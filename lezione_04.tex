
Si assume come criterio di progettazione una caduta di tensione pari ad un massimo di $0.5 \%$
$$
\Delta_{V_{amm}}  = 0.5\% U_n
$$
Il relee direzionale in esame è il 67n, presenta un circuito amperometrico ed uno voltmetrico.
La tensione secondaria è $U_{sr} = 100V$, la prestazione del TV è 50VA, la lunghezza sarà semplice da calcolare, la sezione candidata è più piccola a quella necessaria a circuito amperometrico, $S = 1.5 mm^2$, dunque
$2\rho \frac{L}{S}*0.25^2 \leq 0.5$, la lunghezza massima ricavata è circa 40 metri.

Il posizionamento dei TA e dei TV possono esser einstallati a monte o a valle del dispositivo generale, esistono 6 casi possibili, in funzione della specifica applicazione dei relee.

\begin{enumerate}
    \item TA di tipo induttivo e TV induttivi, sono delicati dunque è opportuno collegarli a valle del dispositivo generale, il toroide impiegato per i guasti a terra è molto robusto, può essere collegato a monte del DG
    \item Analogamente al caso precedente il toroide può essere collegato a valle
    \item TA collegato a monte, comanda direttamente l'apertura del DG, è un vantaggio perchè il guasto potrebbe anche occorrere a monte del DG nonostante la bassa probabilità, lo spostamento a monte del TV prevede la disposizione di un fusibile e un sezionatore
    \item TV di tipo ohmico-induttivo può essere spostato a monte
    \item TO trasformatore omopolare è spostato a monte, non ha criticità
\end{enumerate}
Qualora ci sia il guasto al TV, necessario per il 67n, è necessario in ogni caso utilizzare la logica del 51n sempre presente.

\section{Collaudo di un impianto di terra di cabina}
Logica inerente la protezione dell'impianto di terra, si deve considerare il fattore $r$ per il quale moltiplicare la corrente di guasto per calcolare la corrente dispersa nel terreno, capire perchè il tempo di guasto è quello del distributore e non di utente.

La corrente di guasto $I_F$ in caso di neutro isolato è pari a $I_F = (0.03L_1 + 0.2L_2)U$, qualora la rete sia esercita a neutro isolato. L'estensione della 
rete in MT è nota al distributore che comunicherà questa corrente, c'è il 
progressivo cambio di gestione del neutro, si sta effettuando il passaggio al 
sistema a neutro compensato, si hanno i valori di 50A a 20kV, 40A a 15kV e 75A 
a 30kV.

Il distributore comunica anche il tempo di eliminazione del guasto $t_f$ che può anche essere maggiore di 10 secondi, la corrente è sensibilmente più piccola rispetto al caso di neutro isolato. Il guasto a terra può avvenire a monte del DG, dunque il tempo è quello del distributore ma si richiude comunque una corrente nell'impianto di terra dell'utente.


Per la progettazione dell'impianto di terra bisognerebbe cosiderare effettivamente la corrente dispersa nel terreno $I_E$, in generale collegata alla corrente $I_F$ mediante un coefficiente moltiplicativo minore di 1.

Se il distributore non collega gli schermi a terra, il distributore potrebbe richiedere di non collegare lo schermo a terra, in tal caso il coefficiente $r=1$, la condizione è più conservativa.

Se il distributore collega gli schermi a terra e non comunica nulla all'utente, $r=0.7$, se il numero delle cabine è minore di 3, allora $r=1$.

Nella verifica della bontà di funzionamento dell'impianto di terra, se $R_E$ 
non è minore di $U_{TP}/I_E$ verifica se la rete è magliata, se sì, verifica se 
$R_E\leq 2\frac{U_{TP}}{I_E}$ altrimenti se non è magliata misura le tensioni 
di contatto $U_{T} \leq U_{TP}?$, in caso contrario è necessario asfaltare la 
superficie aumentando la resistività superficiale del terreno. Diminuisce 
drasticamente la tensione di contatto rispetto alla tensione di contatto a 
vuoto.
Dopo aver asfaltato la superficie vanno rieseguite le misure di contato.

Le strutture generali di un impianto di terra partono da un semplice anello  
interrato ad una profondità ottimale di 0.5-1 metri, al massimo solitamente 
80cm, si potrebbero prevedere dei picchetti, in seguito un ulteriore anello 
immerso ad una profondità maggiore, solitamente collegato ad una griglia 
elettrosaldata e collegato al dispersore di stabilimento, ovvero la struttura 
in cemento dello stesso. L'impianto di terra di cabina dunque, previsti tutti 
gli elementi appena citati può essere collegato ulteriormente solo alla terra 
di fondazione dello stabilimento, non ha senso aggiungere ulteriori dispersori 
intenzionali.

Per la protezione contro i guasti a terra è necessario un relee, sicuramente un 
TA toroidale omopolare con il primario passante e un anello toroidale con un 
punto collegato a terra, è il modo più semplice di rilevare il guasto a terra, 
si è visto l'accorgimento da adottare per gli schermi dei cavi, è il 51n, 
potrebbe accadere per esigenze di selettività, qualora l'estensione della rete 
in cavo sia tale da ricorrere all'utilizzo del 67n.

Qualora ci fosse un guasto a terra a monte del DG, deve intervenire la 
protezione dell'ente distributore, la corrente di guasto potrebbe richiudersi 
mediante le capacità a valle del DG, che sentirebbe comunque una corrente di 
guasto e interverrebbe senza necessità, a causa della corrente capacitiva che 
si richiude a valle superando la corrente di taratura.

Qual'è la lunghezza limite oltre il quale la corrente di guasto in un impianto adiacente causa l'intervento di un relee omopolare tarato a 2A?
Il calcolo è abbastanza brutale
$$
I_g = 0.2\cdot L\cdot U \leq 80\% \cdot 2 \leq 1.6 A \Rightarrow L \leq \frac{1.6}{U}\ [m]
$$
Con una rete a 20kV si ha un'estensione di 0.4kmm ovvero 400 metri, a 15kV sono 533m.

Qualora la rete non possa essere limitata in estensione si deve necessariamente prevedere il 67n. La protezioned eve funzionare sia per neutro isolato che neutro compoensato, potrebbe essere cambiato dal gestore per vari motivi di interconnessioni ecc...
\begin{table}[h]\centering
    \begin{tabular}{|c|c|c|}\hline
        Stato del neutro & Protezione 67n & Protezione 51n \\ \hline
        Neutro isolato & NO  & $2A/170ms\ |\ \text{Inutilizzata}$ \\ \hline
        Neutro isolato & SI & Inutilizzata $|\  1.4I_f\ /\ 170ms $ \\ \hline  
        Neutro compensato & NO &  $2A\ /\ 450ms \ |\ 1.4I_F\ /\ 170ms $\\ \hline
        Neutro compensato & SI & Inutilizzata $|\ 1.4I_f\ /\ 170ms$ \\ \hline
    \end{tabular}
\end{table}
Il valore $1.4 I_F$ è molto cautelativa, la corrente di secondo guasto sarà sicuramente molto maggiore.

L'utente può essere considerato connesso ad un impianto di terra globale, un 
sistema interconnesso composto dagli impianti di terra di più cabine, il 
sistema tende ad essere equipotenziale, la progettazione dell'impianto di terra 
si può ritenere implicitamente soddisfatta, va previsto un anello interrato 
intorno alla cabina in ogni caso, il conduttore di terra deve avere una sezione 
di almeno 16$mm^2$ in rame o $50mm^2$ se in acciaio e una resistenza meccanica 
e alla corrosione, per resistere alle sollecitazioni termiche la sezione deve 
essere almeno $25mm^2$, esistono dei coduttori di terra particolarmente 
importanti che vanno dimensionati con più attenzione.

Si utilizza il seguente schema
\begin{itemize}
    \item CT1, conduttore di terra, anche se ha l'ufficio di un conduttore di protezione $>\ 25mm^2$
    \item CT2, conduttore di terra, sufficiente $25mm^2$
    \item PE1, messa a terra del neutro 
    \item PE2, conduttore di protezione 
    \item Sbarra di terra
\end{itemize}
In caso di corto circuito monofase a terra nel quadro di bassa tensione, il PE2 si dimensiona per tale corrente per il tempo di interruzione del guasto dell'interruttore in BT.

In caso di guasto sulla carcassa in BT, sono coinvolti i conduttori PE1 e CT1, la corrente è la stessa di prima ma è cambiato il tempo di intervento,. interviene in questo caso l'interruttore a monte con il tempo ritardato, di conseguenza il dimensionamento di CT1 è più oneroso.

\subsection{Esercizio in parallelo dei trasformatori}
Un sistema molto affidabile è quello radiale doppio, c'è una rigida 
duplicazione dei componenti, ha un problema di costo ma ha il problema 
dell'esercizio in parallelo dei trasformatori, si deve sempre limitare la 
corrente di corto circuito in BT, ad un valore limite di 50kA.

%Schema con congiunntore
In questa configurazione ogni trasformatore è dimensionato per poter alimentare 
integralmente l'impianto, il congiuntore Q3 va chiuso sempre prima di aprire Q1 
o Q2.

Se il numero di trasformatori in parallelo è maggiore di 2 allora anche gli interruttori dei singoli trasformatori sono chiamati ad interrompere una $I_{cc}$ di quella che dovrebbero interrompere se non fossero in parallelo.

\subsection{Ulteriori considerazioni sulla cabina}
La cabina è considerata una costruzione dunque è sottoposta a concessione edilizia, deve rispettare le prescrizioni del regolamento edilizio, la parte relativa al dimensionamento di solai e pavimenti è particolarmente onerosa a causa dei carichi concentrati notevoli causati dal trasformatore. Non si devono propagare gli incendi all'esterno.


