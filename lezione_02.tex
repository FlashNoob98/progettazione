\section{29/11/22}
%Disegno trasformatore triangolo-stella con cc monofase BT
In caso di corto circuito monofase in BT, si ha una corrente monofase in AT che 
impegna solo la fase A e B, si vuole analizzare il fenomeno in termini di terne 
di sequenza
$$
\bar{I}_{ds}  = \bar{I}_{is} = \bar{I}_{0s}
$$
Lato primario dunque si dovrebbe trovare la corrente relativa alla fase C pari 
a zero.
La componente omopolare è blocacta nel triangolo in BT, la componente alla 
sequenza diretta 
$$
\begin{aligned}
	I_{dp} &= \frac{I_{ds}}{m}e^{-j\pi/6} \\
	I_{ip} &= \frac{I_{is}}{m}e^{j\pi/6} \\
	I_{c} &= \alpha I_{dp} + \alpha^2 I_{ip} \\
    I_c &= \alpha \frac{I_{ds}e^{-j\pi/6}}{m} + \frac{\alpha^2 I_{ds}e^{j\pi/6}}{m} = 0
\end{aligned}
$$
L'ultimo termine è pari a zero moltiplicando gli $\alpha$ per i termini 
esponenziali.

Se la cabina è alimentata da un solo trasformatore, non è un problema la non 
selettività delle protezioni lato bassa e lato media, non si avrebbe comunque 
la possibilità di alimentare parzialmente l'impianto.
Nel caso in cui fossero presenti carichi importanti, ad esempio servizi di 
sicurezza alimentati a monte dell'interruttore di BT, che vanno alimentati 
anche in caso di guasto diventa importante il discorso della selettività. 

Se ci sono più trasformatori non in parallelo, ovvero esitono più vie di 
alimentazione dell'impianto, nel caso di CC in BT è fondamentale che intervenga 
l'interruttore in BT relativo a quella sezione di impianto.

Ad esempio si realizza un impianto con tre trasformatori, tre inerruttori a monte di ciascun trasformatore e un interruttore generale a monte di questi tre, si potrebbe anche non avere una completa selettività in caso di intervento del DG e di un interruttore su una delle tre linee, si potrà comunque richiudere il DG dopo una breve discontinuità dell'alimentazione, si può ripristinare il servizio.

\subsection{Esempio}
Si ha una cabina con un TA 300/5, classe di precisione 5p30, l'errore composto è fondamentale per i sistem idi protezione, vengono effettuate delle tarature e tutto deve funzionare con precisione, soprattutto ad esempio con i relee a microprocessore. 5p30 indica che si ha una precisione del 5\% fino a 30 volte la corrente nominale, è un dispositivo difficile da realizzare, deve funzionare in un ampio range di correnti.

Le tarature fornite dal distributore sono le seguenti
\begin{table}[h]\centering
    \begin{tabular}{c c c}
        $I>>$  & 250A & 0,5s\\
        $I>>>$ & 600A & 0,12s
    \end{tabular}
\end{table}

Nella cabina sono presenti due trasformatori in olio da 630kVA con le seguenti caratteristiche e corrente di corto circuito trifase:
$$\begin{aligned}
    &V_p = \\
    &i_{N1} = \\
    &19.2 A \\
    &I_{N2} = 630000/400/\sqrt{3} = 909A \\
    &I_{CC2} = 909/0.06 \\
    &I_{K}' = 15150/50 = 303A
\end{aligned}
$$
Per la corrente di corto circuito bifase minima si ha $0.47\cdot I_k/m$, con
$$
I_k  = \frac{I_r}{\frac{I_r}{I_{kMT}'m}+ \frac{u_{cc}}{100}}
$$
deve essere inferiore a $I_cc=a $

All'atto dell'inserzione si ha una corrente pari a circa 11 volte la corrente nominale.

Il DG è 250/600, deve funzionare tutto in maniera selettivia, si ricerca una selettività energetica. C'è un limite superiore alla taratura. 

Per le protezioni 50

Si comincia a vedere la taratura a 450A , per CC lato BT, si ha una correte richiamata al pimario.


Considerando la curva di $\frac{t_r}{t_i}$ con $=0.65$ si ha $\frac{t_r}{T_i} = 0.1$ per un trasformmatore da 630 kVA si hano 30ms, si è certi che con questa taratura non intervenga il ritardato all'atto dell'inserzione del trasformatore.

In caso di CC bifase lato bassa, la corrente richiamata l primario è $139.1\ A$ certamente interviene il ritardato.

\subsection{CC trifase} In questo caso la $I_K' = 304\ A$ aggiungendo un 30\% a causa di asimmetrie si ha circa 395A, non interviene il relee istantaneo. 

Si rappresenta in tabella il tempo di eliminazione del guasto in funzione della corrente al primario
%%%% Disegna grafico

La $I_K' = 304\ A$, la corrente minima è 139.1 A.


La sezione in BT si proteggge mediante un termometro a contatti e come protezione di rincalzo un interruttore con $I_n = 1250\ A$ data la corrente al secondario pari a 909 A, il relee termico è regolabile, fino ad $1.1I_{tr}$ vista la presenza dell'interruttore, dunque il termico sarà a 1000 A.
Per la taratura del magnetico si sceglie una soglia pari a $6I_{tr}$ ovvero $6000/50 = 120\ A$ con un tempo di eliminazione del guasto di 80ms, c'è la selettività amperometrica fino a 120 A, la differenza tra 0.2 e 0.08 c'è una differenza di 70ms, ovvero una selettività cronometrica, la differenza tra il tempo di interruzione del guasto e il tempo di intervento degli interruttori potrebbe essere inferiore al tempo per il quale il produttore garantisce la protezione. Con una corrente lato BT $> 6500 A$ la selettività cronometrica non è garantita.
Tra gli interruttori in BT e quelli in MT è garantita una selettività di tipo energetico. 

Se la protezione del trasformatore fosse stata affidata al fusibile con corrente nominale $I_n = 40 A$ e $I_3 = 130A \geq 2.5\cdot I_n$.
%PAG 112 tuttonormel

\section{Relee di protezione}
Il sistema di protezione deve essere idoneo e resistente agli errori anche in un certo range di variazione di correnti, i relee di protezione sono di vario tipo: elettromeccanico, statico-elettronico, a microprocessore.
Quelli attualmente più utilizzati sono di tipo statico o a microprocessore.
L'inserimento di queste protezioni sono ad inserzione diretta o indiretta se si trovano a diretto contatto o meno con la linea in MT mediante TA o TV di protezione. La lettera ``p'' nella sigla del TA e del TV indica proprio la protezione.
I dispositivi di protezione hanno spesso bisogno di un'alimentazione per poter funzionare correttamente, fornita mediante fonti esterne o direttamente dai trasformatori di protezione.

I relee solitamente adoperati sono:
\begin{itemize}
    \item Dispositivo termico di protezione (26)
    \item Relee di minima tensione (27)
    \item Relee di massima corrente a tempo inverso (51)
    \item Relee di massima corrente istantaneo (50) e ritardato (51)
    \item Relee di massima corrente omopolare ritardato (51n)
    \item Relee direzionale di massima corrente (67n)
    \item Relee di richiusura in corrente alternata (79) 
\end{itemize}
Prima di mettere in funzione l'impianto è necessario verificare la funzionalità dei dispositivi di protezione.
I relee di massima corrente possono essere collegati su due o tre TA.

Le soglie di intervento del relee a tempo indipendente sono tutte ritardabili, per quanto riguarda invece le caratteristiche a tempo-dipendente esiste la seguente relazione
$$
t  = TMS \frac{K}{\left(\frac{I}{I_s}\right)^\alpha -1}
$$
$TMS$ è il Time-Multiplier-Setting mentre $K$ ed $\alpha$ sono i parametri di taratura.
\begin{table}[h]\centering
    \begin{tabular}{|c | c|c|c|}\hline
            & NIT & VIT & EIT \\ \hline
        $\alpha$ & 0.2 & 1.0 & 2.0 \\ \hline
        $K$ & 0.14 & 13.5 & 80.0 \\ \hline
    \end{tabular}
\end{table}
Le sigle sono NIT(Normal inverse time), VIT(Very inverse time), EIT(Extreme inverse time).

Tutti i sistemi di protezione devono essere molto affidabili ma soprattutto 
soddisfare delle specifiche, il relee di massima corrente ha una prima soglia, 
una seconda e terza soglia di tipo ritardata e istantanea, inoltre deve avere 
un tempo base minore o uguale a 50ms per corrente di ingresso pari ad 1.2$I_n$, 
ovvero il tempo che impiega il relee a misurare la corrente ed emettere il 
comando di apertura. Il tempo di ricaduta ovvero il tempo necessario a 
ripristinare lo stato di polarizzazione dopo aver percepito il guasto, nel caso 
di due relee in serie, l'intervento di un relee a valle deve intervenire prima 
di quello a monte, ovvero quello a monte non deve ricadere nello stato iniziale 
di guasto anche se quello a valle è intervenuto aprendo il circuito.
Dunque il tempo di ricaduta è l'intervallo di tempo tra l'istante di tempo in 
cui si modifica la grandezza misurata e l'istante in cui il relee cambia lo 
stato, ovvero il suo circuito d'uscita cambia lo stato.

Il rapporto di ricaduta rappresenta il rapporto tra il valore della grandezza controllata che determina la diseccitazione della protezione e il valore che ne aveva determinato l'eccitazione.
Più il relee è sensibile e più questo rapporto tende ad 1, è pari a 0.95 nei relee di massima corrente, in un relee di minima tensione il rapporto è invece leggermente maggiore di 1.

\subsection{Esempio due sistemi in serie}
Siano due sistemi di protezione connessi in serie, A e B, di quanto si deve conferire il ritardo per compensare gli errori e le modalità di funzionamento dei due componenti? Il $\Delta t$ di ritardo deve essere superiore al tempo base del relee più l'errore considerato positivo a vantaggio della sicurezza più il tempo di interruzione, è necessario aggiungere anche un tempo di attesa (tempo di ricaduta) affinchè si ritorni nello stato precedente, si aggiunge un margine di sicurezza $\tau$
$$
\Delta t = t_{eB} + \Delta\varepsilon_{teB} + t_{iB} + t_{rA} + \tau
$$
gli ordini di grandezza sono circa 0.1s, 25ms, 0.1s, 40ms, 200ms.
Per il relee queste grandezze sono fondamenali, i circuiti amperometrici devono avere inoltre una sovraccaricabilità per almeno un secondo.

I relee istantanei possono essere dotati di bobina antagonista per bilanciare la seconda armonica dovuta all'inserzione del trasformatore.

Il relee omopolare inoltre è utilizzato per misurare la somma delle correnti, funge da differenziale. 