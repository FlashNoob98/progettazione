
\subsection{fulminazioni dirette}
Un edificio in cemento armato, si assume munito di LPS, è collegato a terra 
mediante un collettore e un dispersore con impedenza $Z$, all'interno 
dell'edificio è presente una massa generica collegata alla linea elettrica a 
sua volta collegata mediante un SPD all'impianto di terra, sono presenti 
inoltre una tubazione idrica e una tubazione gas.

Se le masse esterne sono collegate ad un manicotto isolante più lungo di un 
metro non è necessario collegarle a terra. L'SPD produce una scarica verso 
terra se si supera un certo valore di tensione.

Data la scarica, metà corrente coinvolgerà il dispersore, l'altra metà le 
rimanenti strutture.

I dispersori non possono essere considerati equipotenziali, lo stesso può 
essere visto come una linea, con un modello uniformemente distribuito formato 
da tante resistenze in parallelo, a 50Hz, salendo con la frequenza invece 
diventa un modello di tipo RLC, si parlerà di impedenza convenzionale di terra, 
la corda di terra non sarà un conduttore equipotenziale, si risolverà questo 
problema di equipotenzialità con gli schermi. Nel caso di fulminazione 
indiretta ci saranno sovratensioni inferiori, quella diretta è la più 
problematica.

L'impedenza convenzionale di terra è il rapporto tra il valore di picco della tensione totale di terra e il valore di picco della corrente dispersa.

Esistono due tipologie di accoppiamento induttivo: di tipo A se si considera la 
sovratensione su un conduttore elettricamente collegato alla calata su cui 
avviene la fulminazione, se la fem indotta è tale da configurare un circuito di 
tipo spira, si potrebbe superare la rigidità dielettrica per cui ci sarà una 
circolazione della corrente, il coefficiente di autoinduzione della spira 
comprende la calata.
$$
L_a = l\frac{\mu_0}{2\pi} \ln \frac{d}{l}
$$

Tenendo conto degli ordinari valorid di $d$ ed $l$ il valore di induttanza per unità di lunghezza è pari a 1-1.5$\ \mu H/m$.

È presente un forzamento in corrente, si vuole conoscere la sovratensione:
$$
u_a = L_{au} ll\frac{di}{dt}
$$

Il termine derivata è espresso come $i/T_c$ dove $i$ è il valore di picco della corrente di fulmine,
si suppone che cui siano successivi colpi di fulmine
$$
\frac{di}{dt} = 
$$
è variabile per i primi colpi, si assume 
%stop

$$
T_c = \frac{I}{\text{max}\{ \frac{di}{dt} \}} = 2.5\mu s
$$
la $u$ sarà 
$$
u_a = L_{au} -w-a
$$
l'impedenza convenzionale dunque assume i valori medi pari a 1.25 $\mu H/m$
 per i colpi successivi, se la spira è più lunga di un metro si ha una sovratensionde di 5kV.

 \subsection{Accoppiamento di tipo M}
Si consideri la mutua induzione, tra la calata ed un circuito non collegati tra loro, il valore 5a. Se si vuole calcolare la corrente indotta:
$$
L_m \frac{d_{i}}{dt} = R_a i_n + L_a\frac{d_{i_n}}{dt}
$$
La corrente indotta è molto più piccola rispetto a quella inducente, i valori di $L_m$ sono più piccoli ai precedenti, si trascura il fenomeno di tipo induttivo.
Per limitare le sovratensioni di tipo induttivo va limitata la dimensione delle spire.


\section{Equipotenzialità e schermi}
Siano date due masse $A$ e $B$, collegate mediante un conduttore di protezione 
all'impianto di terra, con la circolazione di una corrente nella corda di terra 
non si può più ritenere che ci sia l'equipotenzialità, esisteranno delle 
sovratensioni a causa della resistenza del conduttore, esisterà comunque un 
accoppiamento induttivo tra i conduttori e le spire, esisterà anche una piccola 
sovratensione trasversale tra i due conduttori attivi, considerata trascurabile 
data la relativa distanza piccola.

Si prevede la presenza di uno schermo, collegato anch'esso alle masse delle 
apparecchiature. Circolerà una certa corrente $I_s$ nello schermo, ma al suo 
interno il campo magnetico sarà nullo dunque la differenza di potenziale tra le 
masse si riduce alla caduta di tensione ohmica sullo schermo $R_s\cdot I_s$, lo 
schermo deve comunque garantire la condizione di campo nullo al suo interno.

Si sono eliminate le sovratensioni di tipo induttivo, si è ridotta la d.d.p.
Si vuole stimare il valore della resistenza dello schermo al fine di avere
$$
R_s\cdot I_s \leq U_w
$$
Imponendo che la ripartizione della corrente mediante una legge lineare di partitore di corrente resistivo si può risolvere facilmente il problema.

$$
I_s = \frac{IR_{ct}}{R_{ct}+R_s} \Rightarrow \frac{R_sR_{ct}I_s}{R_{ct}+R_s} \leq U_W
$$

$$
R_s = \frac{R_{ct}+U_w}{R_{ct}I-U_w}
$$

$$
S = \rho l \left(\frac{I}{U_w}- \frac{1}{R_{ct}} \right)
$$

Se fossero presenti altri conduttori oltre al conduttore di terra, si 
ridurrebbe la corrente nello schermo, se lo schermo fosse già esistente e non 
si potrebbe aumentare la sua sezione. Si garantisce comunque un campo nullo al 
suo interno.

\section{Sovratensione su una linea che entra nell'edificio}
Anche in questo caso la linea sarà già dotata di schermo messo a terra in 
prossimità del ddispersore dell'edificio e messo a terra in prossimità della 
cabina MT/BT.

Si ritiene che in caso di fulminazione dell'edificio il 50\% della corrente si ripartisca tra il dispersore e gli altri corpi metallici, in realtà se si volesse calcolare la corrente che circola nella generica parte metallica
$$
nI_F = \frac{ZI}{\frac{Z_1}{n}+Z} \Rightarrow I_F = \frac{IZ}{Z_1+nZ}
$$
Lo schermo sui conduttori condurrà ``indietro'' la corrente di scarica, la 
sovratensione potrebbe essere talmente elevata da rompere il dielettrico dei 
conduttori che condurranno conseguentemente la corrente di scarica.
$$
I_S = \frac{n_cI_F}{R_c + n_cR_s} \quad I_{condutt} = \frac{R_S}{R_c+nR_s}
$$

La messa a terra del cavo fa sì che le sovratensioni tra i conduttori attivi e la massa siano riconducibili solo alla caduta resistiva sul cavo.

Si può ridurre il valore della sovratensione tra i conduttori attivi e la massa ad un valore conveniente che dipende solo dalla resistenza dello schermo. Occorre tenere conto se lo schermo è collegato in intimo contatto con il terreno o solo all'estremità, nel primo caso la corrente viene dispersa lungo il suo percorso e si considera la lunghezza equivalente pari a 
$$
L_{eq} = 8\sqrt{\rho_t}
$$
dunque la sezione
$$
\frac{\rho 8 \sqrt{\rho_t}I_S}{S} = U_w
$$

\section{Protezioni dalle sovratensioni}
Si adottano vari sistemi come l'SPD ad arrivo linea per sovratensioni di media 
intensità o LPS per sovratensioni di alta entità. In caso di sovratensioni a 
basso contenuto energetico sono previsti solo SPD nei quadri secondari e di 
apparecchiatura.

Le protezioni si dividono in tipo preventivo, più costose come l'LPS, e repressivo, l'opportuno distanziamento dell'LPS dai vari circuiti garantisce un'altra forma di protezione.

Una considerazione preliminare può essere quella di sfruttare la struttura in cemento armato se ritenuta continua, può essere considerata già una struttura LPS, inoltre è importante cercare di perseguire quanto possibile l'equipotenzialità mediante l'adozione di schermature.

Il funzionamento degli SPD si può suddividere in tre fasi:
\begin{enumerate}
    \item Funzionamento ordinario: impedenza elevata e nessun incidenza sul circuito installato
    \item La sua impedenza si riduce a seguito della sovratensione, permette il passaggio di corrente con conseguente abbassamento della sovratensione
    \item Ritorno alla condizione iniziale di funzionamento ordinario
\end{enumerate}

Si definisce la tensione massima continuativa di esercizio deve essere 
superiore del 10\% della tensione nominale, deve sopportare le sovratensioni 
temporanee a frequenza di rete.
Deve sopportare la corrente continuativa di esercizio, deve essere trascurabile 
altrimenti potrebbe causare un malfunzionamento dei differenziali, non 
dovrebbero superare $I_{dn}/3$.

Nella fase 2 c'è la corrente nominale di scarica, il valore di picco della 
corrente con forma d'onda $8/20\ \mu s$. La corrente massima di scarica è il 
valore di picco della massima corrente con forma d'onda che l'SPD può scaricare 
almeno una volta.

La corrente ad impulso, il valore di picco della corrente con forma d'onda 10/
350 ad impulso.

Tensione residua, la tensione ai terminali dell'SPD a seguito della scarica, 
impedisce la persistenza della corrente a seguito della sovratensione, si 
presenterebbe una tensione permanente sulle masse. 

Inoltre si stabilisce il livello di protezione, il livello normalizzato di 
tensione immediatamente superiore alla tensione di innesco e alla tesione 
residua. 

La corrente susseguente nella fase 3, il valore che circola nell'SPD dopo la 
scarica, esistono SPD di tre tipi: a commutazione, a limitazione di tensione, 
combinato.
A commutazione sono gli spinterometri o i diodi controllati, i più diffusi ed 
economici, hanno una tenisone di innesco elevata, il loro livello di protezione 
potrebbe non essere adeguato per proteggere la BT, possono essere adoperati 
sostituendo il gas al loro interno per abbassare il livello di innesco, 
potrebbero non essere in grado di interrompere la corrente susseguente.

SPD a limitazione di tensione come i diodi Zener, non è presente la corrente di corto circuito sostenuta, per il comportamento ai differenziali potrebbero presentare un problema a causa delle correnti continuative diverse da zero.

Gli SPD combinati sono una combinazione dei due, solitamente collegati in serie.
Gli SPD di classe 1 sono adatti per qualunque tipo di circuito, proteggono 
dalla fulminazione diretta. Gli SPD di classe 2 proteggono i quadri di 
distribuzione e di apparecchiatura, quelli di classe 3 hanno un grado di 
protezione più fine, adatti alle apparecchiature elettroniche.

La legge generale per proteggere le apparecchiature è quella di disporre gli 
SPD il più vicino possibili alle apparecchiature da proteggere, esiste infatti 
un fenomeno di propagazione e riflessione della tensione, dal momento in cui 
inizia la tensione cimatrice dell'SPD, la tensione sull'apparecchiatura 
continua comunque a salire.
Le cadute potrebbero essere contemporanee o non contemporanee ad $U_p$ oppure potrebbero presentarsi casi più gravosi, il miglior collegamento è quello di tipo entra-esci.
La sovratensione aggiuntiva dovuta al fenomeno di riflessione è 
$$
2\rho \frac{D}{v}
$$
bisogna dunque ridurre al minimo la distanza tra l'SPD e le apparecchiature da proteggere.

Le sezioni minime dei conduttori di collegamento per gli SPD sono:
\begin{itemize}
    \item 6$mm^2$ per classe I
    \item 6$mm^2$ per classe II 
    \item 1.5$mm^2$ per classe III
\end{itemize}
Se deve scaricare una corrente maggiore di 50kA si usa una sezione di 16$mm^2$, si potrebbero collegare più SPD in cascata, un utile coordinamento prevede quello di far intervenire prima quelli di classe I e così via, si adotta un criterio di energia passante $I^2t$.
È importante anche il coordinamento degli SPD con i differenziali.

