
Il quadro di commutazione di un gruppo elettrogeno, qualora quest'ultimo fosse molto distante, è preferibile spostarlo in 
prossimità dei carichi privilegiati, nelle condizioni di normale funzionamento 
altrimenti ci sarebbe un cavo di collegamento molto lungo, creando uno 
svantaggio duranto le condizioni di normale funzionamento in cui per il 99\% 
del tempo, l'alimentazione proviene dalla rete.

In caso di corto circuito, la tensione di eccitazione potrà diminuire, andrebbe 
spinta al massimo per garantire la stabilità della tensione di trasmissione.
In caso di sovravelocità potrebbero esserci problemi al motore primo, in 
particolare al sistema pompa-iniezione, il generatore invece è protetto 
dall'interruttore automatico del gruppo, c'è un limite al sovraccarico.

L'interruttore del gruppo svolge un ruolo fondamentale alla protezione 
dell'intero sistema, va scelta la corrente di sicuro intervento entro un'ora 
$I_h$, anche il motore può funzionare in sovraccarico con $k_P \sim 1.1$. 
Il motore primo non ha un'intelligenza prevista.
$$
P = \sqrt{3}U_n I_n \cos\varphi \leq k_P
$$
Si ipotizza il rendimento unitario per l'alternatore.
La corrente di sicuro intervento entro un'ora è dunque:
$$
I_n \leq \frac{k_P}{\sqrt{3} U_n \cos \varphi}
$$
con $k_P$ la potenza del motore primo, il $\cos\varphi$ si assume pari a 0.8.
La protezione può diventare insufficiente se il $\cos\varphi>0.8$ e intempestiva se $\cos\varphi<0.8$.

La norma non stabilisce alcunchè per quanto riguarda il sovraccarico 
dell'alternatore che potrebbe invece funzionare in sovraccarico per un tempo 
limitato. Si suppone che possa funzionare per un'ora con una corrente il $10\%$ 
superiore alla nominale.

L'interruttore deve intervenire anche per l'apertura del magnetico, la soglia 
deve essere minore della minima corrente di CC, sicuramente interverrà per una 
corrente più elevata, sarà solitamente pari a tre volte la corrente nominale.

Qualora l'interruttore sia ritardato, anche in presenza del booster, la $I_{cc}$
sarà tre volte la corrente nominale, a causa dei circuiti di sovraeccitazione
questa sarà maggiore.

La scelta del gruppo elettrogeno, dipende dal carico, dalla loro criticità, 
implica l'identificazione della potenza del motore primo e del generatore.
La potenza attiva erogabile dall'alternatore dipende dal $\cos\varphi$ 
assumibile pari a 0.8. Il motore primo dovrà fornire una potenza pari a 
$P=0.8S$ maggiorato di un 10\% dunque $P=0.88S$ con $S$ la potenza apparente 
dell'alternatore.

Se il $\cos\varphi$ diventa pari a 0.9 la potenza del motore primo diventa più 
grande, va in sovraccarico, l'alternatore non può percepire il sovraccarico del 
motore, la corrente è la stessa, dualmente se il $\cos\varphi$ è minore di 0.8, 
a parità di potenza attiva, aumenterà la corrente, va in sovraccarico 
l'alternatore.

Parametri del gruppo elettrogeno
\begin{itemize}
    \item Potenza continua, il valore massimo che può fornire con continuità 
    con carico elettrico costante.
    \item Potenza prima, la potenza massima che può fornire in servizio sempre 
    continuo con carico elettrico variabile. La potenza media non deve essere 
    superiore al 70\% della potenza prima.
    \item Potenza continua in emergenza, deve dare una potenza costante in un 
    intervallo di tempo limitato, 500 ore, con carico costante.
    \item Potenza prima in emergenza, la potenza massima con carico variabile 
    con un intervallo di tempo di 200 ore.
\end{itemize}
Per gli alternatori esiste la potenza nominale continua di base e continua di 
punta.
Le condizioni ambientali devono essere specificate, qualora fossero differenti 
si dovrà procedere al de-rating.

Per la potenza di base bisogna tener conto della potenza del carico senza 
sovrabbondare in maniera scriteriata, causando un inutile aggravio economico e 
per i sistemi di protezione, eventualmente con maggiori perdite di rendimento.
inoltre i carichi devono essere equilibrati al fine di far funzionare 
l'alternatore al meglio. in caso di funzionamento monofase la potenza sarà un 
terzo della potenza totale, non si può comunque superare la corrente massima, 
va comunque fatta la riflessione di un funzionamento degradato.

Un gruppo elettrogeno destinato ad alimentare carichi squilibrati va progettato 
su misura, accordato con il costruttore si sovradimensionerà l'alternatore.

L'ultimo aspetto da considerare è l'alimentazione dei grandi motori asincroni, durante l'avviamento saranno presenti correnti elevate.
$$
I_n = \frac{P_n}{\sqrt{3}U_n\eta\cos\varphi}
$$
ma all'avviamento la corrente è $KI_n$ con $K$ tra 4 e 8, l'alternatore dovrà fornire questa corrente, dunque la potenza pari a
$$
S = \sqrt{3}U_nKI_n
$$
sarebbe un sovradimensionamento eccessivo, si stima invece che l'alternatore 
può essere sovraccaricato di un fattore di 2.5, senza necessità di 
sovradimensionamento, il tempo di avviamento è breve.

Si considera un rendimento di $\eta=0.95$ e $\cos\varphi=0.8$ allora $S=3P$
$$
S = \frac{\sqrt{3} U_n K I_n}{2.5}
$$
Anche il motore primo deve essere preso pari a $3P$ dato che la potenza attiva sarà
$$
P = \sqrt{3} U_n I_n K \cos\varphi_{avv}
$$
il $\cos\varphi$ all'avviamento è solitamente la metà di quello nominale.
Dunque la potenza attiva richiesta al motore primo sarà $KP/2$.

\subsection{Protezione contro le sovracorrenti}
Per la protezione contro i contatti indiretti si sfrutta la teoria delle 
componenti in sequenza.
$$
\begin{aligned}
    &\left.x_d(s)\right|_{s=0} = x_d \\
    &x_d(\infty) = x''_d\\
    &x'_d
\end{aligned}
    $$
La componente simmetrica della corrente di CC ha valore pari ad $I''_K = U_0/x''_d$.
Si definisce l'impedenza base di macchina:
$$
Z_n = \frac{U_n^2}{S} = \frac{U_n}{\sqrt{3}I_n}
$$
dunque 
$$
\frac{I''_k}{I_n} = \frac{U_0}{x''_dI_n} = \frac{U_0\sqrt{3}Z_n}{x''_dU_n}
$$
Olre al valore efficace è importante trovare anche il valore di picco della 
corrente, si assume in maniera conservativa $2\sqrt{2}I''_k$.

Qualora l'alternatore fosse dotato di booster, non ha senso parlare di corto 
circuito, il booster sostiene la corrente di guasto.


Corto circuito di fase, la corrente sarà la radice di 3 diviso due per la $I''_k$ nell'ipotesi legittima che la reattanza alla componente simmetrica sia pari a..
$$
I''_{k2} =\frac{3}{2}I''_k
$$

La corrente di corto circuito monofase sarà più gravosa
$$
I''_{k1} = \frac{3U_0}{x''_d + x_i + x_o} \simeq \frac{3U_0}{2x''_d + x_o}
$$

La $I''_{k1}$ è più grande della $I_{2k}$ di circa $1.3 \simeq 1.4$, lo stesso ragionamento si può distinguere tra la corrente monofase transitoria:
$$
I'_{k1} = \frac{3U_0}{x''_d + x_i +  x_o}
$$'
La conoscenza di $I''_{k1}$ permette di dimensionare l'interruttore al fine di riuscire ad interrompere almeno la corrente di picco.
Va anche valutata la corrente di corto circuito minima che si verifica a fine linea.

Se il neutro non è distribuito, la corrente di $I_CC$ minima si calcola con le 
reti di sequenza,
la $I_{cc}$ minima bisogna riferirsi alla fase transitoria, la corrente di 
corto circuito.

Se il neutro è distribuito ed ha la stessa sezione del conduttore di fase, la 
condizione peggiore è ancora quella di corto circuito bifase transitoria.

La corrente di corto circuito minima è la mminima tra la $I'_{K1}$ e $I'_{K2}$, valutare se incide la minore sezione del neutro.

Se il gruppo elettrogeno è destinato a funzionare in parallelo con la rete, (solitamente i generatori asincroni), si deve verificare che la corrente di corto circuito sulle sbarre in BT non superi il valore di 50 kA.

\subsection{Protezione contro i contatti indiretti}
Se il gruppo deve funzionare in parallelo alla rete allora la tipologia di 
protezione deve essere necessariamente la stessa della rete, se il gruppo è 
destinato a funzioanre in isola in condizioni ordinarie, allora si comporta 
come un sistema con cabina propria, può avere qualsiasi tipo di protezione.

Se nelle condizioni di funzionamento ordinario il sistema è di tipo TT, quando 
si va in emergenza sarà corredato di interruttori differenziali, non c'è la 
rete ma ci sarà il gruppo elettrogeno, il neutro sarà collegato all'impianto di 
terra, di fatto si è configurato un sistema TN.

Si potrebbe gestire come sistema IT, con neutro sezionabile.

