
Si possono inserirre gli SPD a monte ma ciò potrebbe causare problemi per 
quanto riguarda la protezione dai contatti indiretti in caso di 
malfunzionamento dell'SPD, si porterebbero le masse ad una tensione pericolosa 
$U_E$ con una corrente non percepita dal differenziale.
C'è però il vantaggio di proteggere anche il differenziale dalle sovratensioni, 
si ricorre ad una soluzione ingegnosa, per far intervenire la protezione a 
monte si collega il neutro a terra, mediante uno spinterometro a valle, in tal 
modo non si avranno correnti continuative nel circuito di terra ma un eventuale 
guasto degli SPD varistometrici causerà una corrente di guasto elevata che si 
richude nel neutro, causando l'intervento dell'interruttore a monte.

La scelta dell'SPD spinterometrico deve avere una capacità di scarica pari 
almeno alla somma delle capacità di scarica dei tre SPD superiori.

\section{Alimentazione ed illuminazione di emergenza}

È necessario prevedere un'alimentazione di emergenza, si tratterà di 
alimentazione di sicurezza e di riserva, non si può assumere che ci sia sempre 
disponibilità continua dell'energia. È improbabile creare una rete con due 
alimentazioni indipendenti, si deve avere una alimentazione di emergenza per un 
designato gruppo di utenze privilegiate.

L'alimentazione di sicurezza è destinata a garantire la sicurezza delle 
persone, altrimenti ssi parlerà di alimentazione di riserva per motivi diversi.

Quando si parla di assicurare il servizio di sicurezza si comprendono la 
sorgente, i circuiti e gli altri componenti elettrici di sicurezza, dovranno 
avere certe caratteristiche.

Si deve garantire un'alimentazione di assoluta continuità, si utilizzeranno in 
questo caso gli UPS, si parla invece di interruzione brevissima se minore di 0.
15s, breve se maggiore di 0.15s ma minore di 0.5 secondi, media se maggiore di 
0.5 ma minore o uguale a 15 secondi, lunga se maggiore di 15 secondi.


Per quanto riguarda le sorgenti si prevedono batteria di accumulatori, 
generatore indipendente dall'alimentazione ordinaria, oppure si potrebbe 
prevedere una linea indipendente dalla linea ordinaria, è un'utenza molto 
particolare con alimentazione proveniente da due cabine primarie diverse.

L'ubicazione del locale deve essere accessibile solo al personale addestrato, 
bisogna assicurare che non ci siano problemi legati all'incendio, non si devono 
propagare fumi e gas.

Per quanto riguarda i circuiti di sicurezza è opportuno predisporre un circuito 
indipendente da quello ordinario che ha il compito di collegare la sorgente di 
sicurezza e la parte dell'impianto che si desidera alimentare, si deve 
assicurare l'indipendenza dagli altri circuiti, non si devono attraversare 
luoghi con pericolo d'incendio a meno che non siano realizzati con materiali 
che siano resistenti al fuoco, non devono attraversare luoghi con pericolo 
d'esplosione, è consigliabile omettere la protezione contro il sovraccarico, è 
ritenuto più accettabile il rischio contro il sovraccarico piuttosto che la 
disalimentazione del carico.

È preferibile inoltre la protezione dai contatti indiretti senza interrompere 
l'alimentazione, per i corto circuiti è opportuno avere la protezione per 
circuiti distinti per disalimentare il minor numero possibile di servizi.

Durante la vita degli impianti si può avere l'alimentazione mediante servizio 
ordinaria, o esclusivamente dalla sorgente di sicurezza oppure l'utneza sarà 
alimentata normalmente dalla rete ed eccezionalmente dalla rete di sicurezza.

La rete generalmente presenta variazioni di tensioni, interruzione 
dell'alimentazione, sovratensione, variazioni di frequenza, armoniche.

Per le variazioni di frequenza, che possono essere $\pm 5\%$, i sistemi 
elettrici a partire dalla trasmissione fino all'utenza finale sono in cascata.
si ha una progressiva degradazione della tensione, con frequenza wwvariabile.

Si vedono ora le diverse varianti dRLL'ups....

In condizioni normali si alimenta dalla rete ma c'è comunque la carica della 
batteria che può essere al piombo o al Nickel-Cadmio, la configurazione è 
abbastanza semplice ma ha il difetto intrinseco di un tempo di commutazionne 
non nullo.

Un UPS ``Online'' è caratterizzato da assoluta continuità, in condizioni 
normali did funzionamento il flusso proviene da ambo le parti verso la 
batteria, se l'inverter va in sovraccarico per $I>1.5I_n$ automaticamente 
l'alimentazione si sposta direttamente sulla rete, la rete alimenta le utenze e 
la batteria. Si assicura il tempestivo passaggio all'alimentazione della rete.
Si ha continutà assoluta in questo caso.
Nello schema reale è presente un bypasss manuale che permette di isolare 
l'intero UPS, garantisce il funzionamento del sistema collegando direttamente 
le utenze alla rete.

I parametri di scelta di un UPS sono la potenza nominale, la potenza attiva 
(con fattore di poteza circa 0.7 o 0.8), la distorsione armonica, l'autonomia, 
la tensione in uscita.
Si è sempre privilegiato un gruppo di contiutià statico, si possono utilizzare 
gruppi di continuità rotanti, suddivisi in riserva limitata e riserva 
illimitata. Un esempio a riserva illimitata è con l'ausilio di un motore Diesel.
Un'alimentazione di assoluta continuità senza convertitori invece prevede la 
presenza di un motore-volano-generatore-motore diesel.

\subsection{Illuminazione di sicurezza e riserva}
La prima deve permettere l'esodo delle persone in caso di emergenza, quella di 
riserva deve permettere la continuazione delle attività svolte.
Si parla anche di illuminazione sussidiaria, si intende di riserva ma se ne 
prescrive l'esistenza in stabilimitenti o altri luoghi di lavoro per evitare 
situazioni di pericolo.
Per le vie di esodo bisogna garantire un illuminamento minimo di 1lux sulla 
linea mediana e 0.5 lux in una fascia centrale pari alla metà della larghezza 
della via di esodo. Nei luoghi di pubblico spettacolo, l'illuminamento minimo 
ad un metro dal piano di calpestio in corrispondenza di scale e porte deve 
essere 5 lux oppure 2 lux nelle altre zone, l'autonomia deve essere di un'ora.

I corpi illuminanti potrebbero essere distinti da quelli ordinari, si deve 
prevedere un circuito indipendente, si potrebbe per una questione di 
opportunità adoperare gli stessi apparecchi illuminanti, si deve garantire la 
possibilità di alimentarli con due linee distinte.
Un'unità di controllo nell'apparecchio potrebbe tenere sempre in carica la 
batteria, in caso di guasto avverrà la commutazione sulla batteria.

\section{Protezioni con alimentazione alternativa}

Un gruppo elettrogeno ha solitamente una corrente di corto circuito pari a 
circa 5 volte la corrente nominale, ciò è sicuramente inferiore alla corrente 
di corto circuito proveniente dalla rete, ciò potrebbe causare un non 
intervento delle protezioni.

Vengono solitamente utilizzati i generatori sincroni a poli salienti, 
presentano un volano per garantire una stabilità della frequenza, ovvero 
ridurre le variazioni di velocità dovute al motore diesel, inoltre devono avere 
la presenza di un avvolgimento smorzatore.

L'eccitazione d ella macchina deve aumentare a fronte di carico induttivo, èe 
necessario che il riferimento di tensione sul rotore deve essere maggiore alla 
tensione in uscita e proporzionale alla potenza reattiva, dunque alla caduta di 
tensione interna alla macchina, da compensare.

Si deve prevedere la commutazione rete-gruppo, esiste un quadro di commutazione 
che può commprendere un interruttore gruppo, la rete potrebbe ritornare dopo un 
breve intervallo, dunque non va avviato subito il gruppo, inoltre in caso di 
sequenze di avviamenti andranno comunque previste delle pause, tutti questi 
tempi potrebbero raggiungere i 60 secondi, soltanto dopo che è avvenuto 
l'avviamento si chiude l'interruttore del gruppo dando consenso alla 
commutazione.

Se il gruppo funziona in isola ha un regolatore di velocità a statismo 
regolabile, ovvero a velocità fissata, si tende a cercare uno statismo pari a 
zero, ovvero una velocità indipendente dalla potenza richiesta. In caso di 
funzionamento in parallelo con la rete invece la frequenza è imposta e lo 
statismo può essere variato.

