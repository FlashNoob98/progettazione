\PassOptionsToPackage{dvipsnames}{xcolor}
\documentclass[a4paper,11pt]{book}
\usepackage[a4paper, total={7in, 8in}]{geometry}

\usepackage{ucs}
\usepackage[utf8]{inputenc}
\usepackage{amsmath}
\usepackage{amssymb}
\usepackage{siunitx}
\usepackage{cancel}
\usepackage[italian]{babel}
\usepackage{fontenc}
\usepackage{graphicx}
\graphicspath{{img/}}
\usepackage{circuitikz}
\ctikzset{
    resistors/scale=0.7,
    capacitors/scale=0.7,
    inductors/scale=0.7,
    sources/scale=0.7
    }

\usepackage{float}
\usepackage{xcolor}

\usepackage{hyperref}
\hypersetup{
    colorlinks=true,
    linkcolor=black,
}
\usepackage{arydshln} % dashed lines in matrix (array)



%Definizione 'globale' larghezza immagini
\newcommand{\picwid}{0.35\linewidth}
\usepackage{tikz}
\usepackage{pgfplots} % Plot dei grafici
\pgfplotsset{compat = 1.18}
% Riduzione tempi di compilazione esternalizzando le figure in pdf separati
\usepgfplotslibrary{external}
%\usetikzlibrary{pgfplots.external}
%\tikzexternalize[prefix=PDF/]
\usetikzlibrary{patterns} % libreria per disegnare i pattern
%\tikzset{prefix={PDF/}}
\usepackage{pstricks-add}

\usepackage{caption}
\usepackage{subcaption}

\usepackage{polynom}

\newcommand{\Lap}{\ensuremath{\mathcal{L}}} %L di Laplace
\renewcommand{\Re}{\ensuremath{\mathfrak{Re}}} %Parte reale
\renewcommand{\Im}{\ensuremath{\mathfrak{Im}}} %Parte immaginaria

\newcommand{\infint}{\ensuremath{\int_{-\infty}^{+\infty}}} %Integrale da -inf
%a +inf
\newcommand{\kinfsum}{\ensuremath{\sum_{k=-\infty}^{+\infty}}}
\newcommand{\ninfsum}{\ensuremath{\sum_{n=-\infty}^{+\infty}}}

\newcommand{\sinc}{\ensuremath{\text{sinc}}}

\usepackage{diagbox} %pacchetto linea diagonale per tabelle
\usepackage{booktabs} % toprule midrule e bottomrule
\usepackage{steinmetz} %pacchetto per notazione angolo

\usepackage{bodegraph} %pacchetto per diagrammi di Bode

\date{\today}
\title{Appunti Progettazione}
\author{Daniele Olivieri}
\begin{document}
\maketitle
\tableofcontents
\chapter{Protezione della cabina}
Il distributore può richiedere le seguenti tarature minime che l'utente non può superare, allo stesso tempo non può richiedere delle tarature inferiori:
\begin{table}[h]\centering
    \begin{tabular}{| c | c | c |}\hline
        $1^\circ$ soglia& $2^\circ$ soglia (51) & $3^\circ$ soglia (50) \\
        $I>$ & $I>>$ & $I>>>$\\ \hline
        da concordare &250A - 500 ms & 600A- 120ms \\ \hline
    \end{tabular}
\end{table}

Il numero 51 è associato all'interruttore ritardato, compatibile con il generale in MT, il 50 è associato all'automatico, l'interruttore deve avere un potere di interruzione di 12,5 kA. Se c'è una cabina con doppio montante in uscita il distributore ammette che possa non essere utilizzato il dispositivo generale. Si utilizzano come interruttori quelli associati ad ogni montante, aumenta l'affidabilità dato che in caso di guasto lato media su una delle due montanti, potrebbe intervenire il generale a monte a causa della poca selettività.

\section{Protezione degli interruttori di manovra e dei sezionatori}
Per proteggere questi dispositivi da sovraccarico, le cui correnti nominali sono sempre molto elevate, sono protetti da sovraccarico mediante fusibili, se si considera ad esempio un fusibile da 63A lato MT, la corrente minima di fusione è circa $I_3 = 215A$, dunque protegge un interruttore da $I_r = 400A$.

La protezione per il sovraccarico può essere ubicata anche a valle dell'apparecchiatura da proteggere, il dispositivo di protezione del sezioantore lato MT potrebbe essere anche l'interruttore generale di BT, purchè si tenga conto del rapporto di trasformazione. I sezionatori e gli interruttori di manovra sezionatori devono avere una corrente nominale di breve durata uguale o superiore alla corrente di corto circuito nel punto di installazione, inoltre devono avere la durata $t_k$ pari almeno al tempo di eliminazione del guasto.

\section{Protezione dei cavi}
Per quanto riguarda la protezione contro il sovraccarico dei cavi in MT le 
norme non forniscono indicazioni, ciò non vuol dire che non si debba verificare 
la protezione dei cavi, si assume che il fenomeno da controllare sia che la 
sovratemperatura che non debba compromettere la durata di vita del cavo, dunque 
la corrente di taratura $I_{tr}(MT) \leq I_z$. Si potrebbe adoperare anche 
l'interruttore generale di BT per proteggere il cavo di MT, andrà tarato in 
modo che $I_{tr}'\leq I_z$. Sarà necessario soddisfare una serie di condizioni, 
un sistema di disequazioni da soddisfare per far funzionare le cose bene, c'è 
una doppia possibilità di scelta. Il fusibile è solitamente ``antipatico'' per 
proteggere lato MT, ad esempio si vuole proteggere il cavo MT con un 
interruttore di manovra sezionatore con fusibile, ci si aspetta che le cose non 
vadano benissimo, per proteggere il cavo contro il sovraccarico  si deve 
garantre che $2.5 I_r \leq I_z$, la minima corrente di fusione deve essere 
minore della portata del cavo, dunque $I_r \leq 0.4I_z$, il cavo è sfruttato al 
40\%, è una scelta infelice, il fusibile funziona bene contro la protezione da 
corto circuiti.

\section{Protezione dei cavi contro CC}
Il criterio di dimensionamento è sempre stato $I^2 t \leq K^2 S^2$, la sezione del cavo della linea MT era sufficiente se $S\geq 35\ mm^2$, con la corrente di corto circuito trifase 12500A e Inserisci TEMPO%secondi

Si considera un caso di guasto lato BT, con un trasformatore da 2000kVA allora ci si chiede se interviene l'interruttore lato MT, interviene con un ritardo intenzionale di 0.5s, si esegue il calcolo, la corrente di CC va riportata al primario e si applica il criterio $I^2t\leq K^2S^2$, $I_b = 48.1\ kA$ la corrente lato bassa è stata calcolata con:$2000/(\sqrt{3}\cdot 400) =2.88kA/0.06 = 48.1kA $ in AT utilizzando il rapporto di trasformazione $20000/400$ pari a 50 sarà $48.1/50 = 0.962\ kA$.
Si calcola dunque $I^2t\leq K^2S^2 \rightarrow S \geq 4.8\ mm^2$ non è dunque la condizione più onerosa per dimensionare il cavo, la condizione di corto circuito 12.5kA lato MT sarà sicuramente più gravosa nonostante il minor tempo di intervento di $0.1s$.

\section{Protezione del trasformatore}
È il componente cruciale e vitale della cabina, va tenuto conto che il trasformatore può essere dotato di protezioni interne come il relee Buchholz, sensibile al gas che si forma quando si innesca una scarica, l'indicatore del livello dell'olio, il termometro a contatti e la valvola di sovrapressione.

È sufficiente monitorare le condizioni di funzionamento del trasformatore, il relee Buchholz è dotato di dispositivi di allarme e scatto, la condizione di allarme è compresa tra i 100 e i 
la condizione di scatto si ha quando il volume di gas supera i 300 oppure la velocità dell'olio supera un certo livello.
L'indicatore del livello dell'olio è utile ad effettuare una diagnositca real-time.

Il termometro a contatti è utile a onitorare la temperatura nei punti 
(hot-spot) vicini all'avvolgimento, anche questo ha una soglia di allarme e di 
scatto.
Se il trasformatore è sigillato esiste una valvola di sovrappressione in grado 
di rilevare l'aumento di pressione in seguito ad una scarica.

Le protezioni interne del trasformatore assolvono in modo diretto l'ufficio di 
proteggerlo ma nella pratica corrente si sfrutta anche l'interruttore 
automatico generale di BT. Se non ci sono queste protezioni termometriche 
interne (oggi presenti in tutti i trasformatori) allora certamente questo sarà 
la protezione principale per proteggere il trasformatore.

L'interruttore generale di BT va tarato, per fargli eseguire una protezione di 
rincalzo per il trasformatore, ad $1.1I_r$ con $I_r$ la corrente nominale 
secondaria del trasformatore.
Questo valore è scelto perchè già sono presenti le protezioni termometriche, se 
queste ultime fossero assenti invece, la taratura viene effettuata proprio ad $I_r$. La corrente di sicuro intervento entro il tempo convenzionale, 
solitamente 2h, è pari a $1.3 I_{tr}$ la condizione di sovraccarico è prevista 
se il trasformatore è normalmente caricato al 70\% della $I_n$ può essere 
sovraccaricato del $30\%$ proprio per 2h, allora la taratura è perfettamente 
compatibile. Se la condizione di carico è superiore al 70\% si diminuisce $I_{tr}$ fino a 0.9. 

\section{Calcolo della corrente di CC}
Il distributore ha imposto delle tarature minime oltre le quali non si può 
andare oltre, le protezioni lato MT devono funzionare come protezioni di 
rincalzo in caso di malfunzionamento delle funzioni lato BT.
La corrente di corto circuito al secondario è:
\begin{equation}I_K(BT) = \frac{mI_r}{m\frac{u_{cc}}{100}+ \frac{I_r}{I_{kMT}'}}
\label{eq:ICC_BT}
\end{equation}
$I_{kMT}'$ è la corrente di corto circuito trifase minima comunicata dal 
distributore nel punto di connessione, $m$ è il rapporto di trasformazione.
La $I_{kMT}'$ dipende certamente dalla potenza del trasformatore della cabina 
primaria, dall'impedenza della linea, dalla distanza della cabina primaria, 
potrebbe dunque essere inferiore a $12.5\ kA$ tale corrente è fondamentale per 
tarare le protezioni ad intervenire anche con una corrente inferiore a $12.5\ kA$, ad esempio il calcolo precedente prevedeva la corrente di CC lato BT pari 
a 48.1 A con potenza infinita della rete, considerando l'ipotesi di $I_{CC}(AT)=12.5 kA$ invece si ottiene una corrente di corto circuito di $\frac{50\cdot 2.88}{50\cdot0.06+\frac{2.88}{12.5}} = 44.6\ A$, gli interruttori devono 
intervenire dunque per una corrente inferiore a quanto precedentemente stimato.

Per dimostrare la \ref{eq:ICC_BT} si vede come al denominatore siano presenti 
due impedenze relative, $i_{cc} = \frac{1}{x_{cc}} = \frac{I_{kMT}' m }{I_r}$.
La corrente di CC lato BT attraverso l'impedenza relativa del trasformmatore 
$$
i_k = \frac{I_k}{I_r} = \frac{1}{x_{cc} + x_T}
$$
dunque
$$
I_k  = \frac{I_r}{\frac{I_r}{I_{kMT}'m}+ \frac{u_{cc}}{100}}
$$

Dipende dunque dalla corrente indicata dal distributore, la corrente di CC richiamata al primario è $I'_k = \frac{I_k}{m}$.

Si considera di avere il CC lato BT, la protezione MT deve intervenire per la 
minima corrente di corto circuito, in modo ritardato, certamente non è pari 
alla corrente di corto circuito trifase, la corto circuito monofase è la più 
piccola se la protezione è su tutte e tre le fasi, altrimenti la bifase se le 
protezioni sono montate su due fasi.

\subsection{CC monofase}
Si considera la corrente di corto circuito monofase $I_{k1} = I_k$, si asusme 
la corrente al circuito monofase identica a quella CC trafase.
In caso di CC bifase
$$
N_1 I_{k1f}' = N_2 I_{k1} \Rightarrow \frac{I_{k1f}'}{I_{k1}} = \frac{N_2}{N_1} = \frac{1}{m\sqrt{3}}
$$
con $I_{k1f}$ la corrente di fase nel trasformatore
$I_{k1}'= 0.95\frac{1}{\sqrt{3}m}I_k = \frac{0.55 I_k}{m}$
Il fattore 0.95 tiene conto dell'abbassamento di tensione dovuto al corto 
circuito.

\subsection{CC bifase}
La corrente di coro circuito $I_{k2}= \frac{\sqrt{3}}{2}I_k$ con $I_k$ la 
corrente di CC trifase.
Viene richiamata nella linea una corrente pari a $2I_{k2}'$ ma 
$$
I_{k2}' = \frac{1}{\sqrt{3}m} I_{k2} = \frac{\cancel{\sqrt{3}}}{\cancel{\sqrt{3}}m} \frac{I_k}{2}
$$
Se la protezione è montata su tre fasi la corrente minore è quella bifase, 
viceversa se fossero state montate solo due protezioni, non ci sarebbe stato 
intervento in questo caso con una taratura a 0.55 ma sarebbe stata necessaria 
una taratura a $0.47\frac{I_k}{m}$.

Un ulteriore funzionamento intrinseco del trasformatore che causa problemi è il 
fenomeno dell'inserzione del trasformatore che potrebbe generare degli scatti 
intempestivi, affinchè il dispositivo (50) non intervenga, occore che abbia una 
corrente di taratura maggiore di $I_{0i}{\sqrt{2}} \simeq 0.707 I_{0i}$ con $I_{0i}$ la corrente di inserzione, è una condizione facilmente soddisfabile.

Il (51) potrebbe intervenire?
$I_{tr}(51)\geq \frac{I_{0i}}{\sqrt{2}}$ se non fosse sufficiente si dovrebbe 
aggiungere un ritardo intenzionale al fine di soddisfare la curva tabellata del 
ritardo intenzionale $t_r$ rispetto al tempo caratteristico del fenomeno $T_i$ 
in funzione di $\frac{I_{r}(51)}{I_{0i}}$.


\section{29/11/22}
%Disegno trasformatore triangolo-stella con cc monofase BT
In caso di corto circuito monofase in BT, si ha una corrente monofase in AT che 
impegna solo la fase A e B, si vuole analizzare il fenomeno in termini di terne 
di sequenza
$$
\bar{I}_{ds}  = \bar{I}_{is} = \bar{I}_{0s}
$$
Lato primario dunque si dovrebbe trovare la corrente relativa alla fase C pari 
a zero.
La componente omopolare è blocacta nel triangolo in BT, la componente alla 
sequenza diretta 
$$
\begin{aligned}
	I_{dp} &= \frac{I_{ds}}{m}e^{-j\pi/6} \\
	I_{ip} &= \frac{I_{is}}{m}e^{j\pi/6} \\
	I_{c} &= \alpha I_{dp} + \alpha^2 I_{ip} \\
    I_c &= \alpha \frac{I_{ds}e^{-j\pi/6}}{m} + \frac{\alpha^2 I_{ds}e^{j\pi/6}}{m} = 0
\end{aligned}
$$
L'ultimo termine è pari a zero moltiplicando gli $\alpha$ per i termini 
esponenziali.

Se la cabina è alimentata da un solo trasformatore, non è un problema la non 
selettività delle protezioni lato bassa e lato media, non si avrebbe comunque 
la possibilità di alimentare parzialmente l'impianto.
Nel caso in cui fossero presenti carichi importanti, ad esempio servizi di 
sicurezza alimentati a monte dell'interruttore di BT, che vanno alimentati 
anche in caso di guasto diventa importante il discorso della selettività. 

Se ci sono più trasformatori non in parallelo, ovvero esitono più vie di 
alimentazione dell'impianto, nel caso di CC in BT è fondamentale che intervenga 
l'interruttore in BT relativo a quella sezione di impianto.

Ad esempio si realizza un impianto con tre trasformatori, tre inerruttori a monte di ciascun trasformatore e un interruttore generale a monte di questi tre, si potrebbe anche non avere una completa selettività in caso di intervento del DG e di un interruttore su una delle tre linee, si potrà comunque richiudere il DG dopo una breve discontinuità dell'alimentazione, si può ripristinare il servizio.

\subsection{Esempio}
Si ha una cabina con un TA 300/5, classe di precisione 5p30, l'errore composto è fondamentale per i sistem idi protezione, vengono effettuate delle tarature e tutto deve funzionare con precisione, soprattutto ad esempio con i relee a microprocessore. 5p30 indica che si ha una precisione del 5\% fino a 30 volte la corrente nominale, è un dispositivo difficile da realizzare, deve funzionare in un ampio range di correnti.

Le tarature fornite dal distributore sono le seguenti
\begin{table}[h]\centering
    \begin{tabular}{c c c}
        $I>>$  & 250A & 0,5s\\
        $I>>>$ & 600A & 0,12s
    \end{tabular}
\end{table}

Nella cabina sono presenti due trasformatori in olio da 630kVA con le seguenti caratteristiche e corrente di corto circuito trifase:
$$\begin{aligned}
    &V_p = \\
    &i_{N1} = \\
    &19.2 A \\
    &I_{N2} = 630000/400/\sqrt{3} = 909A \\
    &I_{CC2} = 909/0.06 \\
    &I_{K}' = 15150/50 = 303A
\end{aligned}
$$
Per la corrente di corto circuito bifase minima si ha $0.47\cdot I_k/m$, con
$$
I_k  = \frac{I_r}{\frac{I_r}{I_{kMT}'m}+ \frac{u_{cc}}{100}}
$$
deve essere inferiore a $I_cc=a $

All'atto dell'inserzione si ha una corrente pari a circa 11 volte la corrente nominale.

Il DG è 250/600, deve funzionare tutto in maniera selettivia, si ricerca una selettività energetica. C'è un limite superiore alla taratura. 

Per le protezioni 50

Si comincia a vedere la taratura a 450A , per CC lato BT, si ha una correte richiamata al pimario.


Considerando la curva di $\frac{t_r}{t_i}$ con $=0.65$ si ha $\frac{t_r}{T_i} = 0.1$ per un trasformmatore da 630 kVA si hano 30ms, si è certi che con questa taratura non intervenga il ritardato all'atto dell'inserzione del trasformatore.

In caso di CC bifase lato bassa, la corrente richiamata l primario è $139.1\ A$ certamente interviene il ritardato.

\subsection{CC trifase} In questo caso la $I_K' = 304\ A$ aggiungendo un 30\% a causa di asimmetrie si ha circa 395A, non interviene il relee istantaneo. 

Si rappresenta in tabella il tempo di eliminazione del guasto in funzione della corrente al primario
%%%% Disegna grafico

La $I_K' = 304\ A$, la corrente minima è 139.1 A.


La sezione in BT si proteggge mediante un termometro a contatti e come protezione di rincalzo un interruttore con $I_n = 1250\ A$ data la corrente al secondario pari a 909 A, il relee termico è regolabile, fino ad $1.1I_{tr}$ vista la presenza dell'interruttore, dunque il termico sarà a 1000 A.
Per la taratura del magnetico si sceglie una soglia pari a $6I_{tr}$ ovvero $6000/50 = 120\ A$ con un tempo di eliminazione del guasto di 80ms, c'è la selettività amperometrica fino a 120 A, la differenza tra 0.2 e 0.08 c'è una differenza di 70ms, ovvero una selettività cronometrica, la differenza tra il tempo di interruzione del guasto e il tempo di intervento degli interruttori potrebbe essere inferiore al tempo per il quale il produttore garantisce la protezione. Con una corrente lato BT $> 6500 A$ la selettività cronometrica non è garantita.
Tra gli interruttori in BT e quelli in MT è garantita una selettività di tipo energetico. 

Se la protezione del trasformatore fosse stata affidata al fusibile con corrente nominale $I_n = 40 A$ e $I_3 = 130A \geq 2.5\cdot I_n$.
%PAG 112 tuttonormel

\section{Relee di protezione}
Il sistema di protezione deve essere idoneo e resistente agli errori anche in un certo range di variazione di correnti, i relee di protezione sono di vario tipo: elettromeccanico, statico-elettronico, a microprocessore.
Quelli attualmente più utilizzati sono di tipo statico o a microprocessore.
L'inserimento di queste protezioni sono ad inserzione diretta o indiretta se si trovano a diretto contatto o meno con la linea in MT mediante TA o TV di protezione. La lettera ``p'' nella sigla del TA e del TV indica proprio la protezione.
I dispositivi di protezione hanno spesso bisogno di un'alimentazione per poter funzionare correttamente, fornita mediante fonti esterne o direttamente dai trasformatori di protezione.

I relee solitamente adoperati sono:
\begin{itemize}
    \item Dispositivo termico di protezione (26)
    \item Relee di minima tensione (27)
    \item Relee di massima corrente a tempo inverso (51)
    \item Relee di massima corrente istantaneo (50) e ritardato (51)
    \item Relee di massima corrente omopolare ritardato (51n)
    \item Relee direzionale di massima corrente (67n)
    \item Relee di richiusura in corrente alternata (79) 
\end{itemize}
Prima di mettere in funzione l'impianto è necessario verificare la funzionalità dei dispositivi di protezione.
I relee di massima corrente possono essere collegati su due o tre TA.

Le soglie di intervento del relee a tempo indipendente sono tutte ritardabili, per quanto riguarda invece le caratteristiche a tempo-dipendente esiste la seguente relazione
$$
t  = TMS \frac{K}{\left(\frac{I}{I_s}\right)^\alpha -1}
$$
$TMS$ è il Time-Multiplier-Setting mentre $K$ ed $\alpha$ sono i parametri di taratura.
\begin{table}[h]\centering
    \begin{tabular}{|c | c|c|c|}\hline
            & NIT & VIT & EIT \\ \hline
        $\alpha$ & 0.2 & 1.0 & 2.0 \\ \hline
        $K$ & 0.14 & 13.5 & 80.0 \\ \hline
    \end{tabular}
\end{table}
Le sigle sono NIT(Normal inverse time), VIT(Very inverse time), EIT(Extreme inverse time).

Tutti i sistemi di protezione devono essere molto affidabili ma soprattutto 
soddisfare delle specifiche, il relee di massima corrente ha una prima soglia, 
una seconda e terza soglia di tipo ritardata e istantanea, inoltre deve avere 
un tempo base minore o uguale a 50ms per corrente di ingresso pari ad 1.2$I_n$, 
ovvero il tempo che impiega il relee a misurare la corrente ed emettere il 
comando di apertura. Il tempo di ricaduta ovvero il tempo necessario a 
ripristinare lo stato di polarizzazione dopo aver percepito il guasto, nel caso 
di due relee in serie, l'intervento di un relee a valle deve intervenire prima 
di quello a monte, ovvero quello a monte non deve ricadere nello stato iniziale 
di guasto anche se quello a valle è intervenuto aprendo il circuito.
Dunque il tempo di ricaduta è l'intervallo di tempo tra l'istante di tempo in 
cui si modifica la grandezza misurata e l'istante in cui il relee cambia lo 
stato, ovvero il suo circuito d'uscita cambia lo stato.

Il rapporto di ricaduta rappresenta il rapporto tra il valore della grandezza controllata che determina la diseccitazione della protezione e il valore che ne aveva determinato l'eccitazione.
Più il relee è sensibile e più questo rapporto tende ad 1, è pari a 0.95 nei relee di massima corrente, in un relee di minima tensione il rapporto è invece leggermente maggiore di 1.

\subsection{Esempio due sistemi in serie}
Siano due sistemi di protezione connessi in serie, A e B, di quanto si deve conferire il ritardo per compensare gli errori e le modalità di funzionamento dei due componenti? Il $\Delta t$ di ritardo deve essere superiore al tempo base del relee più l'errore considerato positivo a vantaggio della sicurezza più il tempo di interruzione, è necessario aggiungere anche un tempo di attesa (tempo di ricaduta) affinchè si ritorni nello stato precedente, si aggiunge un margine di sicurezza $\tau$
$$
\Delta t = t_{eB} + \Delta\varepsilon_{teB} + t_{iB} + t_{rA} + \tau
$$
gli ordini di grandezza sono circa 0.1s, 25ms, 0.1s, 40ms, 200ms.
Per il relee queste grandezze sono fondamenali, i circuiti amperometrici devono avere inoltre una sovraccaricabilità per almeno un secondo.

I relee istantanei possono essere dotati di bobina antagonista per bilanciare la seconda armonica dovuta all'inserzione del trasformatore.

Il relee omopolare inoltre è utilizzato per misurare la somma delle correnti, funge da differenziale. 

\subsection{01-12 Protezioni per guasto a terra}
Per questa tipologia di guasto è candidato il relee 51n che percepisce la componente omopolare, esiste un problema per quanto riguarda la selettività dei sistemi di protezione, è necessario invece il relee 67n, relee direzionale di terra. È previsto in aggiuntiva al 51n che deve invece sempre essere presente, il 67n potrebbe nonn rilevare il doppio guasto a terra.

Caratteristiche del 51n:
\begin{itemize}
    \item Tempo base $\leq$ 50ms per $1.2 \ I_n$
    \item Tempo di ricaduta $\leq$ 100ms
    \item Rapporto di ricaduta $\geq$ 0.9
    \item Assorbimento amperometrico $\leq 0.2 VA$ per $I_n$ pari ad 1A
    \item Assorbimento voltmetrico $\leq 1 VA$ per $I_n$ pari a $5A$
\end{itemize} 

Il relee direzionale di terra 67n ha le seguenti caratteristiche:
\begin{itemize}
    \item Tempo base $\leq$ 80ms
    \item Tempo di ricaduta $\leq$ 100ms
    \item Rapporto di ricaduta $\geq$ 0.9
    \item Assorbimento amperometrico $\leq 0.2 VA$ per $I_n$ pari ad 1A
    \item Assorbimento voltmetrico $\leq 1 VA$
\end{itemize}
Percepisce con grande sicurezza se il guasto è a monte o a valle, è accoppiato al 51n.

Misurando la differenza di fase tra la tensione omopolare e la corrente omopolare si può determinare se il guasto è avvenuto a monte o a valle del relee.
La commutazione dal 51n al 67n deve avvenire entro 1 secondo.

Si analizza il problema della taratura, il 67n potrebbe essere richiesto per un problema di lunghezza dei cavi in MT lato utente, non si può in ogni caso rinunciare al 51n per il doppio guasto a terra.

I TV e i TA induttinvi hanno una serie di problematiche e sensibilità ai corti circuiti, inoltre i TA di protezione sono difficili da realizzarsi, devono funzionare in un ampio range di correnti, è difficile realizzarli ad alta precisione, è rrelativamente importante nel caso di CC da 12,5kA.

I TA di misura ordinari devono avere un range di funzionamento tra 0.1 ad 1.2 $I_n$, saturano rapidamennte, in modo tale da non danneggiarsi in caso di CC e proteggere gli strumenti di misura.

Per alimentare un relee di protezione un TA di misura sicuramente non è sufficiente, il TA andrebbe subito in saturazione, si perde la linearità al secondario, si rischia il non intervento del sistema di protezione.
I TA destinato ai sistemi di protezione è dunque più difficile da realizzarsi, è necessario collegare a terra i circuiti del secondario del TA con dei cavi di sezione minima di 2.5 $mm^2$ se protetti meccanicamente o $4\ mm^2$ se non protetti meccanicamente, per un problema di sicurezza, per ridurre i disturbi.

Il sistema di protezione va posto ad una certa distanza dal TA, ciò dipende dalla prestazione del TA, le seguenti caratteristiche di targa, esercizio:
\begin{itemize}
    \item $Is2$ = 1,2,5 A
    \item $K_{\phi_2} = \frac{I_{\phi 2}}{I_{S2}}$
    \item Tenisone massima $24 kV$
    \item Potenza massia erogata al secondario senza uscire dalla classe di precisione.
    \item Errore composto compreso tra il 5 e il 10 \%
    \item Fattore di potenza specificato dal costruttore
\end{itemize}
Si vede la classe di isolamento che vede la sovratempeatura massimma per un TA di $ 85 \ K$ di classe B. 


ALF fattore limite di precisione, segue la classe di precisione, ad esempio 5p15 indica che la precisione è mantenuta al 15\% fino ad una corrente che è 15 volte la corrente nominale ovvero
$$
ALF = \frac{I_{p_l}}{I_{p_2}}
$$
per i TA in esame si analizzeranno anche i 5p30 a causa delle elevate correnti in gioco.

per un TA con una corrente nominale primaria di 300 A, con un 30 \% garantisco fino a 9kA ma la CC standardizzata e 12.5 kA, sembra insufficiente, la norma CEI 0-16 dichiara idoneo il TA.

Il fattore limite di precisione garantisce la prestazione nominale ma se il carico secondario è inferiore alla prestazione nominale, l'ALF aumenta.
$$
ALF' = ALF \frac{R_s I^2_{s_2} + (VA)_r}{R_s I^2_s + (VA)_C}
$$
il carico secondario è costituito dai collegamenti e dall'impedenza del relee.
Solitamente un relee assorbe anche meno di 1VA, non bisogna comunque superare la prestazione nominale, il carico `vincolato'.

Il TA deve fronteggiare il CC, bisogna determinare le prestazioni in condizioni critiche, si definisce
\begin{itemize}
    \item La corrente termica di breve durata nominale, il valore efficace della corrente primaria che il TA sopporta per un secondo con il secondario in CC, denominata $I_{th}$.
    \item La corrente nominale dinamica, il valore di cresta (di picco) della corrente primaria massima che il TA sopporta senza danno, solitamente $I_{dyn} = 2.5\cdot I_{th}$.
    \item Corrente termica nominale permanente, con il secondario collegato ad un circuito la cui potenza assorbita corrisponde alla prestazione nominale.
\end{itemize}

\newpage
La bobina di ROGOWESKY?: il nucleo su cui è avvolto il primario è formato da un materiale non magnetico, a fronte di una corrente di ingresso si ha una tensione secondaria proporzionale.
In questo caso non è richiesta la messaa terra del secondario, l'apertura del secondario inoltre non danneggia il TA, come accade invece per quelli di tipo induttivo. Dualmente è critica la condizione di CC per il TV.

Le grandezze caratteristiche di un TA non induttivo sono:
\begin{itemize}
    \item Corrente nominale
    \item Tensione nominale secondaria, proporzionale alla corrente
    \item Classe di precisione, solitamente 3p e 5p, più preciso dei precedenti
    \item Tensione più elevata sopportabile, 24kV
    \item $I_{th}$
    \item $I_{dyn}$
    \item Classe di isolamento in termini di sovratemperatura
\end{itemize}

Come si sceglie un TA? La corrente nominale primaria $I_{dr}$ deve essere scelta in base ad una condizione di regime sinusoidale permanente, pari alla corrente di sovraccarico prevedibile nell'impianto.
In genere si sceglie la $I_{dr}$ da 1 a 1.5 la corrente di taratura da sovraccarico.
La corrente nominale secondaria è tra 1 e 5 A, compatibile con le correnti di ingresso dei relee di protezione.
È importante che la corrente di CC sul primario non superi le correnti prima definite $I_{th}$ e $I_{dyn}$. 
La corrente di intervento di una protezione è espressa in valore relativo rispetto alla corrente nominale primaria, ossia 
$I_{tr}/I_{pr}$.

Nell'ipotesi di linearità se al primario la corrente di intervento è 0.8 volte la corrente nominale, allora anche al secondario la corrente di regolazione della protezione sarà pari a 0.8 volte la corrente nominale della protezione.
La scelta del TA di tipo non induttivo presenta simili caratteristiche,
la corrente nominale deve essre pari alla corrente di sovraccarico e allo stesso modo vengono definite $I_{th}$ e $I_{dyn}$.

I cavetti di collegamento suggeriti dalla norma sono $6mm^2$ per $I_{2n} = 5A$ e $4mm^2$ per $I_{2n} = 1A$.
Si vuole calcolare la lunghezza massima di un cavo di collegamento tra secondario e relee, i cavi hanno delle perdite, si deve tener conto della potenza dissipata sui cavetti di collegamento, si supponga in questo caso una corrente secondaria di $5A$ e la potenza nominale assorbita dalle protezioni è $1VA$, la prestazione nominale è $10VA$.
$$
2\rho \frac{L}{S}I^2_{sr} + 1 = 10 \Rightarrow 2\rho\frac{L}{6}\cdot 25 = 9
$$
la potenza dissipata dal cavetto deve essere al massimo pari a 9W.
La $\rho$ è calcolata a $70^\circ$, si può adoperare un cavo in PVC senza molti problemi, dunque sarà $0.0178 \rightarrow 0.0222$, la lunghezza massima sarà pari a 48m.
Lavorando ad un punto inferiore a quello nominale si aumenta il fattore limite di precisione, può dunque essere comodo ridurre la lunghezza dei cavetti.

\newpage
Requisiti del TA per la protezione generale dell'impianto, sono necessarie caratteristiche importanti, alla luce di quanto presentato, i TA sono considerati autmaticamente idonei quelli con le seguenti caratteristiche:
\begin{itemize}
    \item $300/5$ o il $300/1$
    \item Prestazione $10\ VA$ o $5\ VA$ 
    \item Classe di precisione 5p
    \item Fatore limite nominale di precisione 30
    \item Prestazione effettiva a 5A è $0.4 \Omega$ mentre ad 1A è $5\Omega$
\end{itemize}

Si possono considerare TA di tipo non induttivo con le seguenti caratteristiche:
\begin{itemize}
    \item $I_{pr} \geq 300 A$
    \item Corrente termica di breve durata $I_{th} \geq 12.5 kA$
    \item Tensione noinale secondaria $0.2V$
    \item Rapporto di trasfomazione corrente/tensione $300A/0.2V$
    \item 
    \item 5p30 ?
    \item Corrente nominale dinamica paria a 31.5 kA
\end{itemize}

Quello induttivo va a valle del dispositivo di protezione, quello non induttivo può andare a monte.

Per i TA omopolari è importante che diano correnti al secondario con precisioni accettabili sia in caso di guasto a terra che doppio guasto a terra qualunque sia la condizione di esercizio del neutro.
\begin{itemize}
    \item $I_{th} = 1.2 I_{sr}$
    \item $I_{nthcc} = 12.5 kA$
\end{itemize}


\section{TV}
\subsection{TV di tipo induttivo}
L'ufficio dei TV di tipo induttivo è quello di dare informazioni riguardo la tensione stellata rispetto a terra o omopolare o per misurare le tensioni di fase in caso di guasto a terra con neutro isolato.

Si consideri un TV con rapporto $U_n/100$, un morsetto del secondario è collegato a terra, la tensione su una delle bobine del secondario è a 100V, le bobine del primario sono invece collegate tra la fase A e B e tra B e C.
È prevista la presenza di una resistenza di smorzamento in parallelo al secondario, in caso di guasto, il circuito magnetico del TV, conuna grande induttanza, potrebbe andare in risonanza con le capacità lato linea, la resistenza di smorzamento limita tale fenomeno, il valore di tale resistenza è da richiedere al costruttore del TV. Un TV può alimentare più relee in parallelo purchè non si superi la prestazione.

I requisiti per la protezione di tipo generale sono:
\begin{itemize}
    \item Classe $0.5-3p$ ossia ha una precisione di $0.5p$ tra l'80\% e il 120\% della tensione nominale, altrimenti l'errore, fino al 190\% della $I_n$ si ha un errore del $3\%$.
    \item Prestazione nominale 50VA
    \item Induzione di lavoro $\leq 0.7T$ per evitare condizioni di saturazione
    \item Valore della resistenza di smorzamento da inserire
\end{itemize}


\subsection{TV di tipo  non induttivo}
Sono solitamente partitori resistivi o capacitivi, la classe di precisione è sempre $0.5-3p$.
Fattore di tensione, la tensione nominale primaria per determinare la tensione massima per cui il trasformatore garantisce le prestazioni previste dalla norma, pari a 1.9.



Si assume come criterio di progettazione una caduta di tensione pari ad un massimo di $0.5 \%$
$$
\Delta_{V_{amm}}  = 0.5\% U_n
$$
Il relee direzionale in esame è il 67n, presenta un circuito amperometrico ed uno voltmetrico.
La tensione secondaria è $U_{sr} = 100V$, la prestazione del TV è 50VA, la lunghezza sarà semplice da calcolare, la sezione candidata è più piccola a quella necessaria a circuito amperometrico, $S = 1.5 mm^2$, dunque
$2\rho \frac{L}{S}*0.25^2 \leq 0.5$, la lunghezza massima ricavata è circa 40 metri.

Il posizionamento dei TA e dei TV possono esser einstallati a monte o a valle del dispositivo generale, esistono 6 casi possibili, in funzione della specifica applicazione dei relee.

\begin{enumerate}
    \item TA di tipo induttivo e TV induttivi, sono delicati dunque è opportuno collegarli a valle del dispositivo generale, il toroide impiegato per i guasti a terra è molto robusto, può essere collegato a monte del DG
    \item Analogamente al caso precedente il toroide può essere collegato a valle
    \item TA collegato a monte, comanda direttamente l'apertura del DG, è un vantaggio perchè il guasto potrebbe anche occorrere a monte del DG nonostante la bassa probabilità, lo spostamento a monte del TV prevede la disposizione di un fusibile e un sezionatore
    \item TV di tipo ohmico-induttivo può essere spostato a monte
    \item TO trasformatore omopolare è spostato a monte, non ha criticità
\end{enumerate}
Qualora ci sia il guasto al TV, necessario per il 67n, è necessario in ogni caso utilizzare la logica del 51n sempre presente.

\section{Collaudo di un impianto di terra di cabina}
Logica inerente la protezione dell'impianto di terra, si deve considerare il fattore $r$ per il quale moltiplicare la corrente di guasto per calcolare la corrente dispersa nel terreno, capire perchè il tempo di guasto è quello del distributore e non di utente.

La corrente di guasto $I_F$ in caso di neutro isolato è pari a $I_F = (0.03L_1 + 0.2L_2)U$, qualora la rete sia esercita a neutro isolato. L'estensione della 
rete in MT è nota al distributore che comunicherà questa corrente, c'è il 
progressivo cambio di gestione del neutro, si sta effettuando il passaggio al 
sistema a neutro compensato, si hanno i valori di 50A a 20kV, 40A a 15kV e 75A 
a 30kV.

Il distributore comunica anche il tempo di eliminazione del guasto $t_f$ che può anche essere maggiore di 10 secondi, la corrente è sensibilmente più piccola rispetto al caso di neutro isolato. Il guasto a terra può avvenire a monte del DG, dunque il tempo è quello del distributore ma si richiude comunque una corrente nell'impianto di terra dell'utente.


Per la progettazione dell'impianto di terra bisognerebbe cosiderare effettivamente la corrente dispersa nel terreno $I_E$, in generale collegata alla corrente $I_F$ mediante un coefficiente moltiplicativo minore di 1.

Se il distributore non collega gli schermi a terra, il distributore potrebbe richiedere di non collegare lo schermo a terra, in tal caso il coefficiente $r=1$, la condizione è più conservativa.

Se il distributore collega gli schermi a terra e non comunica nulla all'utente, $r=0.7$, se il numero delle cabine è minore di 3, allora $r=1$.

Nella verifica della bontà di funzionamento dell'impianto di terra, se $R_E$ 
non è minore di $U_{TP}/I_E$ verifica se la rete è magliata, se sì, verifica se 
$R_E\leq 2\frac{U_{TP}}{I_E}$ altrimenti se non è magliata misura le tensioni 
di contatto $U_{T} \leq U_{TP}?$, in caso contrario è necessario asfaltare la 
superficie aumentando la resistività superficiale del terreno. Diminuisce 
drasticamente la tensione di contatto rispetto alla tensione di contatto a 
vuoto.
Dopo aver asfaltato la superficie vanno rieseguite le misure di contato.

Le strutture generali di un impianto di terra partono da un semplice anello  
interrato ad una profondità ottimale di 0.5-1 metri, al massimo solitamente 
80cm, si potrebbero prevedere dei picchetti, in seguito un ulteriore anello 
immerso ad una profondità maggiore, solitamente collegato ad una griglia 
elettrosaldata e collegato al dispersore di stabilimento, ovvero la struttura 
in cemento dello stesso. L'impianto di terra di cabina dunque, previsti tutti 
gli elementi appena citati può essere collegato ulteriormente solo alla terra 
di fondazione dello stabilimento, non ha senso aggiungere ulteriori dispersori 
intenzionali.

Per la protezione contro i guasti a terra è necessario un relee, sicuramente un 
TA toroidale omopolare con il primario passante e un anello toroidale con un 
punto collegato a terra, è il modo più semplice di rilevare il guasto a terra, 
si è visto l'accorgimento da adottare per gli schermi dei cavi, è il 51n, 
potrebbe accadere per esigenze di selettività, qualora l'estensione della rete 
in cavo sia tale da ricorrere all'utilizzo del 67n.

Qualora ci fosse un guasto a terra a monte del DG, deve intervenire la 
protezione dell'ente distributore, la corrente di guasto potrebbe richiudersi 
mediante le capacità a valle del DG, che sentirebbe comunque una corrente di 
guasto e interverrebbe senza necessità, a causa della corrente capacitiva che 
si richiude a valle superando la corrente di taratura.


%
Il solaio di una cabina deve resistere a carichi concentrati notevoli, 
l'ingresso dei cavi ai quadri invece può avvenire dal basso e poi risalire, 
occorre predisporre cunicoli e tubazioni nel pavimento, in caso di numerosi 
cavi è previsto un pavimento galleggiante, devono inoltre essere predisposte 
delle distanze minime da rispettare per garantire il passaggio del personale 
addetto, un'altezza minima è quella di 2 metri, oppure si considera un'altezza 
maggiore della diagonale dell'involucro del quadro per permetterne un facile 
spostamento. La larghezza minima deve essere di 80cm per permettere il 
passaggio del personale, deve essere garantito un illumminamento di 200 lux.

Una ulteriore soluzione praticata per utilizzare più trasformatori è quella di 
una struttura ad anello aperto, ha una minore lunghezza dei cavi, dunque un 
costo inferiore rispetto al radiale doppio con congiuntore NA (normalmente 
aperto).

\subsection{Ventilazione delle cabine}
Una ventilazione che funziona male comporta rischi di incendio, c'è un problema 
di riscaldamento in cabina. Si riconosce un forzamento pari alla potenza termica dissipata dalle macchine.
$$
P_t = (P_{fe} + P_{Cu}) = P_0 + P_k\frac{I^2}{I^2_r}\cdot 1.15
$$
le perdite vengono maggiorate del 15\% per tenere conto di tutti gli altri 
elementi che producono calore in cabina, stimato in maniera conservativa.

Si supponga di avere quattro aperture distanti $h$ in altezza, l'area della singola finestratura sarà
$$
A = 0.119 \frac{P_t}{\sqrt{h}}
$$
ricavata assumendo una temperatura interna di $40^\circ$ ed esterna pari a $30^\circ$, se si tiene conto delle reti protettive e gli ostacoli che possono ostruire le aperture, l'area va aumentata del 15\%.

Se le aperture sono sulla stessa parete, il coefficiente raddoppia:
$$
A = 0.238 \frac{P_t}{\sqrt{h}}
$$
si deve supporre che il trasformmatore sia vicino alla parete con le aperture.

La ventilazione naturale viene adoperata fino ad un $30\sim50\%$ del carico, dopodichè viene attivata la ventilazione forzata, se la temperatura supera i $35^\circ$, sostituisce completamente la ventilazione naturale. 
La portata si calcola nel seguente modo:
$$
q_v = 346 P_t
$$
con $q_v$ la portata d'aria espressa in $m^3/h$ da garantire, $P_t$ la potenza 
termica dissipata.

Si deve evitare però che si alzi polvere con una velocità dell'aria maggiore di $3m/s$, i vari componenti elettrici potrebbero perdere la lora tenuta, si contaminerebbero le superfici isolanti.
Si consideri ad esempio un locale sul livello del mare $6m\times 6m$ e altezza $h=3.5m$, sono installati due trasformatori con i seguenti dati 
$$
P_0 = 1.1\quad P_k = 7.6\ kW,\qquad P_t = 1.15(1.1+1.9)\cdot 2 = 6.9 kW
$$
l'apertura con distanza di $h=2.6m$ sarà pari a $A = 0.238\cdot\frac{6.9}{\sqrt{2.6}}= 1.02m^2$ con convezione naturale, approssimata a $1.2m^2$, è prevista un'apertura $2m\times 0.6m$.

La venilazione forzata sarà pari a $$
8.7\cdot 2\cdot 1.15 =  20.01 kW
$$
dunque la portata
$$
q_v = 246\cdot 20.01s = 346 P_t
$$
la velocità invece
$$
v = \frac{q_v}{3600\cdot 2\cdot 1.02}
$$
$1.02$ per tenere conto delle reti, $0.94m/s$, di gran lunga inferiore al limite di $3m/s$ impsoto.

\section{Esempio di cabina}
Si supponga un trasformatore in olio di potenza attiva 450kW e $\cos\varphi$ = 
0.9, dunque una potenza apparente di 500kVA, si considera dunque un 
trasformatore di 600kVA.

Il cavo di collegamento avrà una sezione di $95 mm^2$ in accordo con le indicazioni del dsitributore, il quadro MT avrà una tensione nominale pari a 24kV, tensione di prova 50kV 5Hz, prova a impul



Il sezionatore ha una corrente nominale da 630A, tempo di interruzione 70ms.

I TA scelti sono della classe 5p30 con rapporto di trasformazione 300/5, corrente $I_{dyn} = 31.5\ kA$ e un TA toroidale.
Il fattore limite è rispettato, aumenta a 100 dato che la potenza a  ssorbita dai cavetti e il relee è bassa.

Il cavo di alimentazione del trasformatore MT è posato nel vano di fondazione della cabina, ha una portata di 179A, \SI{35}{\milli\meter^2}

La corrente nominale primaria è 24.2A, la $I_{CC}$ trifase secondaria è 15.2kA.
La $I_{cc}$ secondaria riportata al primario è pari a 405A.

La conduttura che va dal trasformatore al quadro generale deve avere una portata superiore alla corrente nominale del trasformatore, si sceglie un cavo
$$
3(2\times 1\times 300) + 2 + 1\times(150)
$$

Il cavo in BT ha una corrente di breve durata di 25kA, l'interruttore magnetico generale, va tartato.

\dots


\section{Sovratensioni}
Per la portezione da sovratensioni, soprattutto d itipo atmosferico, vanno dimensionati gli SPD.

Una sovratensione è un valore di tensione che va oltre il valore di picco della 
massima tensione nelle normalifunzionamento. Le sovratensioni si possono 
inoltre classificare in funzione della loro durata, breve, media lunga e 
lunghissima durata.

Quando le sovratensioni si hanno tra conduttori attivi, si parla di sovratensioni trasversali, se si hanno tra i conduttori e la terra si parla di sovratensioni di modo comune o longitudinali.

Un'ulteriore classificazione è effetuabile in funzione dell'origine della 
sovratensione, qualora esse siano originate da manovreo  guasti si dicono di 
origine interna, invece le fulminazioni sono di origine esterna.
Le sovratensioni in genere non superano i 2.5kV, la forma d'onda
$
1.2/50\mu s
$
è assunta come forma rappresentativa delle sovratensioni di manovra con tempo 
all'emivalore di 50 $\mu s$, la forma d'onda della corrente di scarica allunga 
i tempi, la forma d'onda è $8/20 \mu s$, sale più lentamente a causa 
dell'induttanza.
Per quanto riguarda la massima sovratensione temporanea a frequenza di rete per il sistema in BT conseguente da un guasto in BT, il valore è pari a 500V per 5 secondi, per il sistema TT era per discernere se raggiungere il limite di 500V.
Si può dedurre che tenendo conto che l'impianto di terra la $R_E$ è maggiore di $5-10 \Omega$ si può affermare che in tal caso, se la sovratensione è 500V e la resistenza di terra $>10\Omega$, invece le persone vengono colpite ...

Si formano più scariche, la prima con carica più elevata e tempo discarica maggiore, le successive più brevi, in generale il fulmine lo si modella con un generatore di corrente ideale, la norma afferma che posso caratterizzare il fulmine con ill tempo al picco e il tempo all'emivalore.

10/350 è la prima forma d'onda, 1/200 il primo colpo di fulmine negativo, 0.25/100 $\mu second$ il successivo?

Per i fenomeni induttivi i colpi finali sono quelli più gravosi, nonostante la 
carica inferiore, i primi colpi hanno un contenuto energetico elevato ma $di/dt$ modesto, dualmente i colpi successivi un contenuto energetico modesto ma 
$di/dt$ elevato. Quando un fulmine colpisce un edificio si parla di 
fulminazione diretta, altrimenti indiretta se non viene investito direttamente.

Possono esserci fenomeni di accoppiamento di tipo induttivo o capacitivo, se la 
sovratensione investe il dispersore di terra, la corrente incontrerà 
un'impedenza convenzionale del dispersore, dunque si produrrà una sovratensione 
causata da un accoppiamento resistivo, le sovratensioni conseguenti avranno la 
stessa forma della corrente di fulmine ovvero 10/350 e sono le più elevate. Le correnti possono raggiungere le decine di kA, nel caso di fulminazione diretta si può considerare che la corrente in ciascun conduttore non superi 10kA.

La circolazione di corrente produrrà un campo magnetico variabile che indurrà una forza elettromotrice, si parla di accoppiamenti di tipo A e accoppiamenti di tipo M, le entità delle sovratensioni di tipo induttivo sono più pericolosi perchè i valori delle pendenze nelle fasi successive, chiamate colpi di ritorno, sono più elevate, si considera la forma 0.25/100 $\mu s$ la forma delle correnti invece saranno $8/20\mu s$, in generale le correnti di scarica non superano i valori di 3 kA.


Per accoppiamento capacitivo invece si può ritenere trascurabile rispetto ai precedenti.

Un'apparecchiatura deve resistere alle sovratensioni, devono avere una tensione di tenuta ed un livello di immunità, ovvero il valore di tensione oltre il quale c'è la perdita.

Tra il livello di immunità e il livello di tenuta è presente un danneggiamento dell'apparecchiatura.

Le sovratensioni per accoppiamento resistivo possono provocare danni, quelle di tipo induttivo invece sono caratterizzare da un'enrgia modesta e possono arrecare danni solo alle apparecchiature.


Sono presenti 4 zone per la bassa tensione, le apparecchiature più sensibili hanno una tensione di tenuta di 1.5kV, nella zona 1....


%
\subsection{fulminazioni dirette}
Un edificio in cemento armato, si assume munito di LPS, è collegato a terra 
mediante un collettore e un dispersore con impedenza $Z$, all'interno 
dell'edificio è presente una massa generica collegata alla linea elettrica a 
sua volta collegata mediante un SPD all'impianto di terra, sono presenti 
inoltre una tubazione idrica e una tubazione gas.

Se le masse esterne sono collegate ad un manicotto isolante più lungo di un 
metro non è necessario collegarle a terra. L'SPD produce una scarica verso 
terra se si supera un certo valore di tensione.

Data la scarica, metà corrente coinvolgerà il dispersore, l'altra metà le 
rimanenti strutture.

I dispersori non possono essere considerati equipotenziali, lo stesso può 
essere visto come una linea, con un modello uniformemente distribuito formato 
da tante resistenze in parallelo, a 50Hz, salendo con la frequenza invece 
diventa un modello di tipo RLC, si parlerà di impedenza convenzionale di terra, 
la corda di terra non sarà un conduttore equipotenziale, si risolverà questo 
problema di equipotenzialità con gli schermi. Nel caso di fulminazione 
indiretta ci saranno sovratensioni inferiori, quella diretta è la più 
problematica.

L'impedenza convenzionale di terra è il rapporto tra il valore di picco della tensione totale di terra e il valore di picco della corrente dispersa.

Esistono due tipologie di accoppiamento induttivo: di tipo A se si considera la 
sovratensione su un conduttore elettricamente collegato alla calata su cui 
avviene la fulminazione, se la fem indotta è tale da configurare un circuito di 
tipo spira, si potrebbe superare la rigidità dielettrica per cui ci sarà una 
circolazione della corrente, il coefficiente di autoinduzione della spira 
comprende la calata.
$$
L_a = l\frac{\mu_0}{2\pi} \ln \frac{d}{l}
$$

Tenendo conto degli ordinari valorid di $d$ ed $l$ il valore di induttanza per unità di lunghezza è pari a 1-1.5$\ \mu H/m$.

È presente un forzamento in corrente, si vuole conoscere la sovratensione:
$$
u_a = L_{au} ll\frac{di}{dt}
$$

Il termine derivata è espresso come $i/T_c$ dove $i$ è il valore di picco della corrente di fulmine,
si suppone che cui siano successivi colpi di fulmine
$$
\frac{di}{dt} = 
$$
è variabile per i primi colpi, si assume 
%stop

$$
T_c = \frac{I}{\text{max}\{ \frac{di}{dt} \}} = 2.5\mu s
$$
la $u$ sarà 
$$
u_a = L_{au} -w-a
$$
l'impedenza convenzionale dunque assume i valori medi pari a 1.25 $\mu H/m$
 per i colpi successivi, se la spira è più lunga di un metro si ha una sovratensionde di 5kV.

 \subsection{Accoppiamento di tipo M}
Si consideri la mutua induzione, tra la calata ed un circuito non collegati tra loro, il valore 5a. Se si vuole calcolare la corrente indotta:
$$
L_m \frac{d_{i}}{dt} = R_a i_n + L_a\frac{d_{i_n}}{dt}
$$
La corrente indotta è molto più piccola rispetto a quella inducente, i valori di $L_m$ sono più piccoli ai precedenti, si trascura il fenomeno di tipo induttivo.
Per limitare le sovratensioni di tipo induttivo va limitata la dimensione delle spire.


\section{Equipotenzialità e schermi}
Siano date due masse $A$ e $B$, collegate mediante un conduttore di protezione 
all'impianto di terra, con la circolazione di una corrente nella corda di terra 
non si può più ritenere che ci sia l'equipotenzialità, esisteranno delle 
sovratensioni a causa della resistenza del conduttore, esisterà comunque un 
accoppiamento induttivo tra i conduttori e le spire, esisterà anche una piccola 
sovratensione trasversale tra i due conduttori attivi, considerata trascurabile 
data la relativa distanza piccola.

Si prevede la presenza di uno schermo, collegato anch'esso alle masse delle 
apparecchiature. Circolerà una certa corrente $I_s$ nello schermo, ma al suo 
interno il campo magnetico sarà nullo dunque la differenza di potenziale tra le 
masse si riduce alla caduta di tensione ohmica sullo schermo $R_s\cdot I_s$, lo 
schermo deve comunque garantire la condizione di campo nullo al suo interno.

Si sono eliminate le sovratensioni di tipo induttivo, si è ridotta la d.d.p.
Si vuole stimare il valore della resistenza dello schermo al fine di avere
$$
R_s\cdot I_s \leq U_w
$$
Imponendo che la ripartizione della corrente mediante una legge lineare di partitore di corrente resistivo si può risolvere facilmente il problema.

$$
I_s = \frac{IR_{ct}}{R_{ct}+R_s} \Rightarrow \frac{R_sR_{ct}I_s}{R_{ct}+R_s} \leq U_W
$$

$$
R_s = \frac{R_{ct}+U_w}{R_{ct}I-U_w}
$$

$$
S = \rho l \left(\frac{I}{U_w}- \frac{1}{R_{ct}} \right)
$$

Se fossero presenti altri conduttori oltre al conduttore di terra, si 
ridurrebbe la corrente nello schermo, se lo schermo fosse già esistente e non 
si potrebbe aumentare la sua sezione. Si garantisce comunque un campo nullo al 
suo interno.

\section{Sovratensione su una linea che entra nell'edificio}
Anche in questo caso la linea sarà già dotata di schermo messo a terra in 
prossimità del ddispersore dell'edificio e messo a terra in prossimità della 
cabina MT/BT.

Si ritiene che in caso di fulminazione dell'edificio il 50\% della corrente si ripartisca tra il dispersore e gli altri corpi metallici, in realtà se si volesse calcolare la corrente che circola nella generica parte metallica
$$
nI_F = \frac{ZI}{\frac{Z_1}{n}+Z} \Rightarrow I_F = \frac{IZ}{Z_1+nZ}
$$
Lo schermo sui conduttori condurrà ``indietro'' la corrente di scarica, la 
sovratensione potrebbe essere talmente elevata da rompere il dielettrico dei 
conduttori che condurranno conseguentemente la corrente di scarica.
$$
I_S = \frac{n_cI_F}{R_c + n_cR_s} \quad I_{condutt} = \frac{R_S}{R_c+nR_s}
$$

La messa a terra del cavo fa sì che le sovratensioni tra i conduttori attivi e la massa siano riconducibili solo alla caduta resistiva sul cavo.

Si può ridurre il valore della sovratensione tra i conduttori attivi e la massa ad un valore conveniente che dipende solo dalla resistenza dello schermo. Occorre tenere conto se lo schermo è collegato in intimo contatto con il terreno o solo all'estremità, nel primo caso la corrente viene dispersa lungo il suo percorso e si considera la lunghezza equivalente pari a 
$$
L_{eq} = 8\sqrt{\rho_t}
$$
dunque la sezione
$$
\frac{\rho 8 \sqrt{\rho_t}I_S}{S} = U_w
$$

\section{Protezioni dalle sovratensioni}
Si adottano vari sistemi come l'SPD ad arrivo linea per sovratensioni di media 
intensità o LPS per sovratensioni di alta entità. In caso di sovratensioni a 
basso contenuto energetico sono previsti solo SPD nei quadri secondari e di 
apparecchiatura.

Le protezioni si dividono in tipo preventivo, più costose come l'LPS, e repressivo, l'opportuno distanziamento dell'LPS dai vari circuiti garantisce un'altra forma di protezione.

Una considerazione preliminare può essere quella di sfruttare la struttura in cemento armato se ritenuta continua, può essere considerata già una struttura LPS, inoltre è importante cercare di perseguire quanto possibile l'equipotenzialità mediante l'adozione di schermature.

Il funzionamento degli SPD si può suddividere in tre fasi:
\begin{enumerate}
    \item Funzionamento ordinario: impedenza elevata e nessun incidenza sul circuito installato
    \item La sua impedenza si riduce a seguito della sovratensione, permette il passaggio di corrente con conseguente abbassamento della sovratensione
    \item Ritorno alla condizione iniziale di funzionamento ordinario
\end{enumerate}

Si definisce la tensione massima continuativa di esercizio deve essere 
superiore del 10\% della tensione nominale, deve sopportare le sovratensioni 
temporanee a frequenza di rete.
Deve sopportare la corrente continuativa di esercizio, deve essere trascurabile 
altrimenti potrebbe causare un malfunzionamento dei differenziali, non 
dovrebbero superare $I_{dn}/3$.

Nella fase 2 c'è la corrente nominale di scarica, il valore di picco della 
corrente con forma d'onda $8/20\ \mu s$. La corrente massima di scarica è il 
valore di picco della massima corrente con forma d'onda che l'SPD può scaricare 
almeno una volta.

La corrente ad impulso, il valore di picco della corrente con forma d'onda 10/
350 ad impulso.

Tensione residua, la tensione ai terminali dell'SPD a seguito della scarica, 
impedisce la persistenza della corrente a seguito della sovratensione, si 
presenterebbe una tensione permanente sulle masse. 

Inoltre si stabilisce il livello di protezione, il livello normalizzato di 
tensione immediatamente superiore alla tensione di innesco e alla tesione 
residua. 

La corrente susseguente nella fase 3, il valore che circola nell'SPD dopo la 
scarica, esistono SPD di tre tipi: a commutazione, a limitazione di tensione, 
combinato.
A commutazione sono gli spinterometri o i diodi controllati, i più diffusi ed 
economici, hanno una tenisone di innesco elevata, il loro livello di protezione 
potrebbe non essere adeguato per proteggere la BT, possono essere adoperati 
sostituendo il gas al loro interno per abbassare il livello di innesco, 
potrebbero non essere in grado di interrompere la corrente susseguente.

SPD a limitazione di tensione come i diodi Zener, non è presente la corrente di corto circuito sostenuta, per il comportamento ai differenziali potrebbero presentare un problema a causa delle correnti continuative diverse da zero.

Gli SPD combinati sono una combinazione dei due, solitamente collegati in serie.
Gli SPD di classe 1 sono adatti per qualunque tipo di circuito, proteggono 
dalla fulminazione diretta. Gli SPD di classe 2 proteggono i quadri di 
distribuzione e di apparecchiatura, quelli di classe 3 hanno un grado di 
protezione più fine, adatti alle apparecchiature elettroniche.

La legge generale per proteggere le apparecchiature è quella di disporre gli 
SPD il più vicino possibili alle apparecchiature da proteggere, esiste infatti 
un fenomeno di propagazione e riflessione della tensione, dal momento in cui 
inizia la tensione cimatrice dell'SPD, la tensione sull'apparecchiatura 
continua comunque a salire.
Le cadute potrebbero essere contemporanee o non contemporanee ad $U_p$ oppure potrebbero presentarsi casi più gravosi, il miglior collegamento è quello di tipo entra-esci.
La sovratensione aggiuntiva dovuta al fenomeno di riflessione è 
$$
2\rho \frac{D}{v}
$$
bisogna dunque ridurre al minimo la distanza tra l'SPD e le apparecchiature da proteggere.

Le sezioni minime dei conduttori di collegamento per gli SPD sono:
\begin{itemize}
    \item 6$mm^2$ per classe I
    \item 6$mm^2$ per classe II 
    \item 1.5$mm^2$ per classe III
\end{itemize}
Se deve scaricare una corrente maggiore di 50kA si usa una sezione di 16$mm^2$, si potrebbero collegare più SPD in cascata, un utile coordinamento prevede quello di far intervenire prima quelli di classe I e così via, si adotta un criterio di energia passante $I^2t$.
È importante anche il coordinamento degli SPD con i differenziali.


%
Si possono inserirre gli SPD a monte ma ciò potrebbe causare problemi per 
quanto riguarda la protezione dai contatti indiretti in caso di 
malfunzionamento dell'SPD, si porterebbero le masse ad una tensione pericolosa 
$U_E$ con una corrente non percepita dal differenziale.
C'è però il vantaggio di proteggere anche il differenziale dalle sovratensioni, 
si ricorre ad una soluzione ingegnosa, per far intervenire la protezione a 
monte si collega il neutro a terra, mediante uno spinterometro a valle, in tal 
modo non si avranno correnti continuative nel circuito di terra ma un eventuale 
guasto degli SPD varistometrici causerà una corrente di guasto elevata che si 
richude nel neutro, causando l'intervento dell'interruttore a monte.

La scelta dell'SPD spinterometrico deve avere una capacità di scarica pari 
almeno alla somma delle capacità di scarica dei tre SPD superiori.

\section{Alimentazione ed illuminazione di emergenza}

È necessario prevedere un'alimentazione di emergenza, si tratterà di 
alimentazione di sicurezza e di riserva, non si può assumere che ci sia sempre 
disponibilità continua dell'energia. È improbabile creare una rete con due 
alimentazioni indipendenti, si deve avere una alimentazione di emergenza per un 
designato gruppo di utenze privilegiate.

L'alimentazione di sicurezza è destinata a garantire la sicurezza delle 
persone, altrimenti ssi parlerà di alimentazione di riserva per motivi diversi.

Quando si parla di assicurare il servizio di sicurezza si comprendono la 
sorgente, i circuiti e gli altri componenti elettrici di sicurezza, dovranno 
avere certe caratteristiche.

Si deve garantire un'alimentazione di assoluta continuità, si utilizzeranno in 
questo caso gli UPS, si parla invece di interruzione brevissima se minore di 0.
15s, breve se maggiore di 0.15s ma minore di 0.5 secondi, media se maggiore di 
0.5 ma minore o uguale a 15 secondi, lunga se maggiore di 15 secondi.


Per quanto riguarda le sorgenti si prevedono batteria di accumulatori, 
generatore indipendente dall'alimentazione ordinaria, oppure si potrebbe 
prevedere una linea indipendente dalla linea ordinaria, è un'utenza molto 
particolare con alimentazione proveniente da due cabine primarie diverse.

L'ubicazione del locale deve essere accessibile solo al personale addestrato, 
bisogna assicurare che non ci siano problemi legati all'incendio, non si devono 
propagare fumi e gas.

Per quanto riguarda i circuiti di sicurezza è opportuno predisporre un circuito 
indipendente da quello ordinario che ha il compito di collegare la sorgente di 
sicurezza e la parte dell'impianto che si desidera alimentare, si deve 
assicurare l'indipendenza dagli altri circuiti, non si devono attraversare 
luoghi con pericolo d'incendio a meno che non siano realizzati con materiali 
che siano resistenti al fuoco, non devono attraversare luoghi con pericolo 
d'esplosione, è consigliabile omettere la protezione contro il sovraccarico, è 
ritenuto più accettabile il rischio contro il sovraccarico piuttosto che la 
disalimentazione del carico.

È preferibile inoltre la protezione dai contatti indiretti senza interrompere 
l'alimentazione, per i corto circuiti è opportuno avere la protezione per 
circuiti distinti per disalimentare il minor numero possibile di servizi.

Durante la vita degli impianti si può avere l'alimentazione mediante servizio 
ordinaria, o esclusivamente dalla sorgente di sicurezza oppure l'utneza sarà 
alimentata normalmente dalla rete ed eccezionalmente dalla rete di sicurezza.

La rete generalmente presenta variazioni di tensioni, interruzione 
dell'alimentazione, sovratensione, variazioni di frequenza, armoniche.

Per le variazioni di frequenza, che possono essere $\pm 5\%$, i sistemi 
elettrici a partire dalla trasmissione fino all'utenza finale sono in cascata.
si ha una progressiva degradazione della tensione, con frequenza wwvariabile.

Si vedono ora le diverse varianti dRLL'ups....

In condizioni normali si alimenta dalla rete ma c'è comunque la carica della 
batteria che può essere al piombo o al Nickel-Cadmio, la configurazione è 
abbastanza semplice ma ha il difetto intrinseco di un tempo di commutazionne 
non nullo.

Un UPS ``Online'' è caratterizzato da assoluta continuità, in condizioni 
normali did funzionamento il flusso proviene da ambo le parti verso la 
batteria, se l'inverter va in sovraccarico per $I>1.5I_n$ automaticamente 
l'alimentazione si sposta direttamente sulla rete, la rete alimenta le utenze e 
la batteria. Si assicura il tempestivo passaggio all'alimentazione della rete.
Si ha continutà assoluta in questo caso.
Nello schema reale è presente un bypasss manuale che permette di isolare 
l'intero UPS, garantisce il funzionamento del sistema collegando direttamente 
le utenze alla rete.

I parametri di scelta di un UPS sono la potenza nominale, la potenza attiva 
(con fattore di poteza circa 0.7 o 0.8), la distorsione armonica, l'autonomia, 
la tensione in uscita.
Si è sempre privilegiato un gruppo di contiutià statico, si possono utilizzare 
gruppi di continuità rotanti, suddivisi in riserva limitata e riserva 
illimitata. Un esempio a riserva illimitata è con l'ausilio di un motore Diesel.
Un'alimentazione di assoluta continuità senza convertitori invece prevede la 
presenza di un motore-volano-generatore-motore diesel.

\subsection{Illuminazione di sicurezza e riserva}
La prima deve permettere l'esodo delle persone in caso di emergenza, quella di 
riserva deve permettere la continuazione delle attività svolte.
Si parla anche di illuminazione sussidiaria, si intende di riserva ma se ne 
prescrive l'esistenza in stabilimitenti o altri luoghi di lavoro per evitare 
situazioni di pericolo.
Per le vie di esodo bisogna garantire un illuminamento minimo di 1lux sulla 
linea mediana e 0.5 lux in una fascia centrale pari alla metà della larghezza 
della via di esodo. Nei luoghi di pubblico spettacolo, l'illuminamento minimo 
ad un metro dal piano di calpestio in corrispondenza di scale e porte deve 
essere 5 lux oppure 2 lux nelle altre zone, l'autonomia deve essere di un'ora.

I corpi illuminanti potrebbero essere distinti da quelli ordinari, si deve 
prevedere un circuito indipendente, si potrebbe per una questione di 
opportunità adoperare gli stessi apparecchi illuminanti, si deve garantire la 
possibilità di alimentarli con due linee distinte.
Un'unità di controllo nell'apparecchio potrebbe tenere sempre in carica la 
batteria, in caso di guasto avverrà la commutazione sulla batteria.

\section{Protezioni con alimentazione alternativa}

Un gruppo elettrogeno ha solitamente una corrente di corto circuito pari a 
circa 5 volte la corrente nominale, ciò è sicuramente inferiore alla corrente 
di corto circuito proveniente dalla rete, ciò potrebbe causare un non 
intervento delle protezioni.

Vengono solitamente utilizzati i generatori sincroni a poli salienti, 
presentano un volano per garantire una stabilità della frequenza, ovvero 
ridurre le variazioni di velocità dovute al motore diesel, inoltre devono avere 
la presenza di un avvolgimento smorzatore.

L'eccitazione d ella macchina deve aumentare a fronte di carico induttivo, èe 
necessario che il riferimento di tensione sul rotore deve essere maggiore alla 
tensione in uscita e proporzionale alla potenza reattiva, dunque alla caduta di 
tensione interna alla macchina, da compensare.

Si deve prevedere la commutazione rete-gruppo, esiste un quadro di commutazione 
che può commprendere un interruttore gruppo, la rete potrebbe ritornare dopo un 
breve intervallo, dunque non va avviato subito il gruppo, inoltre in caso di 
sequenze di avviamenti andranno comunque previste delle pause, tutti questi 
tempi potrebbero raggiungere i 60 secondi, soltanto dopo che è avvenuto 
l'avviamento si chiude l'interruttore del gruppo dando consenso alla 
commutazione.

Se il gruppo funziona in isola ha un regolatore di velocità a statismo 
regolabile, ovvero a velocità fissata, si tende a cercare uno statismo pari a 
zero, ovvero una velocità indipendente dalla potenza richiesta. In caso di 
funzionamento in parallelo con la rete invece la frequenza è imposta e lo 
statismo può essere variato.


%
Il quadro di commutazione di un gruppo elettrogeno, qualora quest'ultimo fosse molto distante, è preferibile spostarlo in 
prossimità dei carichi privilegiati, nelle condizioni di normale funzionamento 
altrimenti ci sarebbe un cavo di collegamento molto lungo, creando uno 
svantaggio duranto le condizioni di normale funzionamento in cui per il 99\% 
del tempo, l'alimentazione proviene dalla rete.

In caso di corto circuito, la tensione di eccitazione potrà diminuire, andrebbe 
spinta al massimo per garantire la stabilità della tensione di trasmissione.
In caso di sovravelocità potrebbero esserci problemi al motore primo, in 
particolare al sistema pompa-iniezione, il generatore invece è protetto 
dall'interruttore automatico del gruppo, c'è un limite al sovraccarico.

L'interruttore del gruppo svolge un ruolo fondamentale alla protezione 
dell'intero sistema, va scelta la corrente di sicuro intervento entro un'ora 
$I_h$, anche il motore può funzionare in sovraccarico con $k_P \sim 1.1$. 
Il motore primo non ha un'intelligenza prevista.
$$
P = \sqrt{3}U_n I_n \cos\varphi \leq k_P
$$
Si ipotizza il rendimento unitario per l'alternatore.
La corrente di sicuro intervento entro un'ora è dunque:
$$
I_n \leq \frac{k_P}{\sqrt{3} U_n \cos \varphi}
$$
con $k_P$ la potenza del motore primo, il $\cos\varphi$ si assume pari a 0.8.
La protezione può diventare insufficiente se il $\cos\varphi>0.8$ e intempestiva se $\cos\varphi<0.8$.

La norma non stabilisce alcunchè per quanto riguarda il sovraccarico 
dell'alternatore che potrebbe invece funzionare in sovraccarico per un tempo 
limitato. Si suppone che possa funzionare per un'ora con una corrente il $10\%$ 
superiore alla nominale.

L'interruttore deve intervenire anche per l'apertura del magnetico, la soglia 
deve essere minore della minima corrente di CC, sicuramente interverrà per una 
corrente più elevata, sarà solitamente pari a tre volte la corrente nominale.

Qualora l'interruttore sia ritardato, anche in presenza del booster, la $I_{cc}$
sarà tre volte la corrente nominale, a causa dei circuiti di sovraeccitazione
questa sarà maggiore.

La scelta del gruppo elettrogeno, dipende dal carico, dalla loro criticità, 
implica l'identificazione della potenza del motore primo e del generatore.
La potenza attiva erogabile dall'alternatore dipende dal $\cos\varphi$ 
assumibile pari a 0.8. Il motore primo dovrà fornire una potenza pari a 
$P=0.8S$ maggiorato di un 10\% dunque $P=0.88S$ con $S$ la potenza apparente 
dell'alternatore.

Se il $\cos\varphi$ diventa pari a 0.9 la potenza del motore primo diventa più 
grande, va in sovraccarico, l'alternatore non può percepire il sovraccarico del 
motore, la corrente è la stessa, dualmente se il $\cos\varphi$ è minore di 0.8, 
a parità di potenza attiva, aumenterà la corrente, va in sovraccarico 
l'alternatore.

Parametri del gruppo elettrogeno
\begin{itemize}
    \item Potenza continua, il valore massimo che può fornire con continuità 
    con carico elettrico costante.
    \item Potenza prima, la potenza massima che può fornire in servizio sempre 
    continuo con carico elettrico variabile. La potenza media non deve essere 
    superiore al 70\% della potenza prima.
    \item Potenza continua in emergenza, deve dare una potenza costante in un 
    intervallo di tempo limitato, 500 ore, con carico costante.
    \item Potenza prima in emergenza, la potenza massima con carico variabile 
    con un intervallo di tempo di 200 ore.
\end{itemize}
Per gli alternatori esiste la potenza nominale continua di base e continua di 
punta.
Le condizioni ambientali devono essere specificate, qualora fossero differenti 
si dovrà procedere al de-rating.

Per la potenza di base bisogna tener conto della potenza del carico senza 
sovrabbondare in maniera scriteriata, causando un inutile aggravio economico e 
per i sistemi di protezione, eventualmente con maggiori perdite di rendimento.
inoltre i carichi devono essere equilibrati al fine di far funzionare 
l'alternatore al meglio. in caso di funzionamento monofase la potenza sarà un 
terzo della potenza totale, non si può comunque superare la corrente massima, 
va comunque fatta la riflessione di un funzionamento degradato.

Un gruppo elettrogeno destinato ad alimentare carichi squilibrati va progettato 
su misura, accordato con il costruttore si sovradimensionerà l'alternatore.

L'ultimo aspetto da considerare è l'alimentazione dei grandi motori asincroni, durante l'avviamento saranno presenti correnti elevate.
$$
I_n = \frac{P_n}{\sqrt{3}U_n\eta\cos\varphi}
$$
ma all'avviamento la corrente è $KI_n$ con $K$ tra 4 e 8, l'alternatore dovrà fornire questa corrente, dunque la potenza pari a
$$
S = \sqrt{3}U_nKI_n
$$
sarebbe un sovradimensionamento eccessivo, si stima invece che l'alternatore 
può essere sovraccaricato di un fattore di 2.5, senza necessità di 
sovradimensionamento, il tempo di avviamento è breve.

Si considera un rendimento di $\eta=0.95$ e $\cos\varphi=0.8$ allora $S=3P$
$$
S = \frac{\sqrt{3} U_n K I_n}{2.5}
$$
Anche il motore primo deve essere preso pari a $3P$ dato che la potenza attiva sarà
$$
P = \sqrt{3} U_n I_n K \cos\varphi_{avv}
$$
il $\cos\varphi$ all'avviamento è solitamente la metà di quello nominale.
Dunque la potenza attiva richiesta al motore primo sarà $KP/2$.

\subsection{Protezione contro le sovracorrenti}
Per la protezione contro i contatti indiretti si sfrutta la teoria delle 
componenti in sequenza.
$$
\begin{aligned}
    &\left.x_d(s)\right|_{s=0} = x_d \\
    &x_d(\infty) = x''_d\\
    &x'_d
\end{aligned}
    $$
La componente simmetrica della corrente di CC ha valore pari ad $I''_K = U_0/x''_d$.
Si definisce l'impedenza base di macchina:
$$
Z_n = \frac{U_n^2}{S} = \frac{U_n}{\sqrt{3}I_n}
$$
dunque 
$$
\frac{I''_k}{I_n} = \frac{U_0}{x''_dI_n} = \frac{U_0\sqrt{3}Z_n}{x''_dU_n}
$$
Olre al valore efficace è importante trovare anche il valore di picco della 
corrente, si assume in maniera conservativa $2\sqrt{2}I''_k$.

Qualora l'alternatore fosse dotato di booster, non ha senso parlare di corto 
circuito, il booster sostiene la corrente di guasto.


Corto circuito di fase, la corrente sarà la radice di 3 diviso due per la $I''_k$ nell'ipotesi legittima che la reattanza alla componente simmetrica sia pari a..
$$
I''_{k2} =\frac{3}{2}I''_k
$$

La corrente di corto circuito monofase sarà più gravosa
$$
I''_{k1} = \frac{3U_0}{x''_d + x_i + x_o} \simeq \frac{3U_0}{2x''_d + x_o}
$$

La $I''_{k1}$ è più grande della $I_{2k}$ di circa $1.3 \simeq 1.4$, lo stesso ragionamento si può distinguere tra la corrente monofase transitoria:
$$
I'_{k1} = \frac{3U_0}{x''_d + x_i +  x_o}
$$'
La conoscenza di $I''_{k1}$ permette di dimensionare l'interruttore al fine di riuscire ad interrompere almeno la corrente di picco.
Va anche valutata la corrente di corto circuito minima che si verifica a fine linea.

Se il neutro non è distribuito, la corrente di $I_CC$ minima si calcola con le 
reti di sequenza,
la $I_{cc}$ minima bisogna riferirsi alla fase transitoria, la corrente di 
corto circuito.

Se il neutro è distribuito ed ha la stessa sezione del conduttore di fase, la 
condizione peggiore è ancora quella di corto circuito bifase transitoria.

La corrente di corto circuito minima è la mminima tra la $I'_{K1}$ e $I'_{K2}$, valutare se incide la minore sezione del neutro.

Se il gruppo elettrogeno è destinato a funzionare in parallelo con la rete, (solitamente i generatori asincroni), si deve verificare che la corrente di corto circuito sulle sbarre in BT non superi il valore di 50 kA.

\subsection{Protezione contro i contatti indiretti}
Se il gruppo deve funzionare in parallelo alla rete allora la tipologia di 
protezione deve essere necessariamente la stessa della rete, se il gruppo è 
destinato a funzioanre in isola in condizioni ordinarie, allora si comporta 
come un sistema con cabina propria, può avere qualsiasi tipo di protezione.

Se nelle condizioni di funzionamento ordinario il sistema è di tipo TT, quando 
si va in emergenza sarà corredato di interruttori differenziali, non c'è la 
rete ma ci sarà il gruppo elettrogeno, il neutro sarà collegato all'impianto di 
terra, di fatto si è configurato un sistema TN.

Si potrebbe gestire come sistema IT, con neutro sezionabile.


% \include{lezione_09}
% \include{lezione_10}
%\include{lezione_11}
%\include{lezione_12}
%\include{lezione_13}
%\include{lezione_14}
%\include{lezione_15}
%\include{lezione_16}

%\include{lezione_17}

%\include{lezione_18}
%\include{lezione_19}
%\include{lezione_20}
% \include{lezione_21}
% \include{lezione_22}
% \include{lezione_23}
% \include{lezione_24}
% \include{lezione_25}
\end{document}
